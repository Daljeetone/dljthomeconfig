%%%%%%%%%%%%%%%%%%%%%%%%%%%%%%%%%%%%%%%%%%%%%%%%%%%%%%%%%%%%%%%%%%%%%%%%%%%%%%
%%                DO NOT CONVERT TABS TO SPACES IN THIS FILE                %%
%%                DO NOT CONVERT TABS TO SPACES IN THIS FILE                %%
%%                DO NOT CONVERT TABS TO SPACES IN THIS FILE                %%
%%%%%%%%%%%%%%%%%%%%%%%%%%%%%%%%%%%%%%%%%%%%%%%%%%%%%%%%%%%%%%%%%%%%%%%%%%%%%%
%% Vim settings                                                             %%
%%    set noexpandtab tw=0 indentexpr=                                      %%
%%    set spelllang=en spellfile=SmokeyManual.en.utf-8.add                  %%
%%    set spell                                                             %%
%%%%%%%%%%%%%%%%%%%%%%%%%%%%%%%%%%%%%%%%%%%%%%%%%%%%%%%%%%%%%%%%%%%%%%%%%%%%%%

%%%%%%%%%%%%%%%%%%%%%%%%%%%%%%%%%%%%%%%%%%%%%%%%%%%%%%%%%%%%%%%%%%%%%%%%%%%%%%
%% TODO:
%%   - Consider putting things in {figure} and and {table} environments so
%%     that TeX can reflow them better
%%   - Reconsider the bullets when nesting {Property} {Definition} {itemize}
%%   - Create a better \NameRef that uses numbers for sections/subsections
%%     but titles for subsubsections
%%   - Replace \ComputerMouse with a bear symbol (CA state flag style)
%%%%%%%%%%%%%%%%%%%%%%%%%%%%%%%%%%%%%%%%%%%%%%%%%%%%%%%%%%%%%%%%%%%%%%%%%%%%%%

\documentclass[11pt]{article}

\newcommand{\newlengthval}[2]{%
	\newlength{#1}%
	\setlength{#1}{#2}%
}

% Body text (only) is indented by a small amount
\newlengthval{\BodyIndentLeft}{0.75in}
\newlengthval{\TransverseMargin}{1in}
\newlengthval{\TopMargin}{1.125in}
\newlengthval{\BottomMargin}{1in}

\usepackage[letterpaper]{geometry}
\newlengthval{\LeftMargin}{\TransverseMargin}
\addtolength{\LeftMargin}{\BodyIndentLeft}
\geometry{top=\TopMargin, bottom=\BottomMargin, left=\LeftMargin, right=\TransverseMargin}

\usepackage{fontspec}
\setmainfont%
	[%
		% Fix en- and em-dashes
		Ligatures=TeX,%
		% Myriad Set is Apple's official font
		% http://istweb.apple.com/software/apps/myriad_set/
		BoldFont=MyriadSetBold.ttf,%
		ItalicFont=MyriadSetTextItalic.ttf,%
		BoldItalicFont=MyriadSetBoldItalic.ttf%
	]%
	{MyriadSetText.ttf}
% Menlo has bold and italic variants
\setmonofont[Scale=MatchLowercase]{Menlo}
% Unicode symbols
\newcommand{\LeftArrowHook}{\char"21A9}
\newcommand{\RightArrowHook}{\char"21AA}
\newcommand{\AppleLogo}{\char"F8FF}
% Other symbols
\usepackage{marvosym}
\usepackage{ifsym}
\newcommand{\Campfire}{\textifsymbol{12}}
\newfontfamily{\WingdingsFamily}{Wingdings.ttf}
\newcommand{\FloppySym}{{\WingdingsFamily\symbol{60}}}
\newcommand{\ClosedFolderSym}{{\WingdingsFamily\symbol{48}}}
\newcommand{\OpenFolderSym}{{\WingdingsFamily\symbol{49}}}
\newfontfamily{\KeyCapsFamily}{Mac Key Caps.ttf}
\newcommand{\KeyCapsCtrlD}{{\KeyCapsFamily DPd}}
\newcommand{\KeyCapsCtrlC}{{\KeyCapsFamily DPc}}
\newcommand{\KeyCapsEnter}{{\KeyCapsFamily R}}

\usepackage[table]{xcolor}
% http://web.apple.com/areas/marketing/corpid/docs/Apple_Type_Guidelines_112213.pdf
% Pantone 429, CMYK 0/0/0/40, RGB 171/175/180
\definecolor{AppleGray}{RGB}{171,175,180}
\definecolor{SmokeyDarkBlue}{RGB}{0,100,185}
\definecolor{SmokeyBlue}{RGB}{28,122,193}
\definecolor{SmokeyLightBlue}{RGB}{212,223,238}

\usepackage{titlesec}
% Sections create new pages
\newcommand{\sectionbreak}{\clearpage}
% Enlarge \subsection and \subsubsection text
% Put \section and \subsection at the same indentation
% Indent \subsubsection from \subsection inwards by \BodyIndentLeft
% Outdent everything by \BodyIndentLeft
\newlength{\TitleWidth}
\newcommand{\Outdented}[1]{\settowidth{\TitleWidth}{#1}\hspace{-\TitleWidth}#1}
\newlengthval{\SectionNumberSep}{0.25in}
% Use \textcolor instead of \color because the latter allows LaTeX to
% introduce a page break immediately after the title
\titleformat{\section}%
	{\bf\Huge}%
	{\Outdented{\textcolor{SmokeyDarkBlue}{\thesection}\hspace{\SectionNumberSep}}}%
	{0in}%
	{\textcolor{SmokeyDarkBlue}}
\titleformat{\subsection}%
	{\bf\Large}%
	{\Outdented{\textcolor{SmokeyDarkBlue}{\thesubsection}\hspace{\SectionNumberSep}}}%
	{0in}%
	{\textcolor{SmokeyDarkBlue}}
\titleformat{\subsubsection}%
	{\bf\large}%
	{}%
	{0in}%
	{\textcolor{SmokeyDarkBlue}}
\titlespacing{\subsubsection}%
	{0in}%
	{\parskip}%
	{0in}
% 1.2x line spacing
%\linespread{1.2}

\usepackage{fancyhdr}
% Notice on left; Page number on right
% Need \mbox to fix vertical alignment between logo and page number
\lhead{\mbox{\color{AppleGray}\AppleLogo} Apple Confidential --- Internal Use Only}
\chead{}
\rhead{\thepage}
%\renewcommand{\headrulewidth}{0in}
% No footer
\lfoot{}
\cfoot{}
\rfoot{}
%\renewcommand{\footrulewidth}{1pt}
% Suppress warning about \headheight being too small
\setlength{\headheight}{1.5\baselineskip}
\pagestyle{fancy}

\usepackage[hang,flushmargin,bottom]{footmisc}
%\renewcommand{\footnoterule}{}

% Block paragraph format
\usepackage{parskip}

\usepackage{hyperref}
\newcommand{\NumNameRef}[1]{\ref{#1}\,\textit{\nameref{#1}}}
\newcommand{\NameRef}[1]{\textit{\nameref{#1}}}

\usepackage{tikz}
\usetikzlibrary{decorations}
\usetikzlibrary{trees}

\pgfdeclaredecoration{complete sines}{initial}{
	\state{initial}
		[
			width=+0pt,
			next state=sine,
			persistent precomputation={
				\pgfmathsetmacro\matchinglength{
					\pgfdecoratedinputsegmentlength / int(\pgfdecoratedinputsegmentlength/\pgfdecorationsegmentlength)
				}
				\setlength{\pgfdecorationsegmentlength}{\matchinglength pt}
			}
		]
		{}
	\state{sine}[width=\pgfdecorationsegmentlength]{
		\pgfpathsine{\pgfpoint{0.25\pgfdecorationsegmentlength}{0.5\pgfdecorationsegmentamplitude}}
		\pgfpathcosine{\pgfpoint{0.25\pgfdecorationsegmentlength}{-0.5\pgfdecorationsegmentamplitude}}
		\pgfpathsine{\pgfpoint{0.25\pgfdecorationsegmentlength}{-0.5\pgfdecorationsegmentamplitude}}
		\pgfpathcosine{\pgfpoint{0.25\pgfdecorationsegmentlength}{0.5\pgfdecorationsegmentamplitude}}
	}
	\state{final}{}
}

% http://tex.stackexchange.com/questions/89129/changing-the-indentation-of-a-tikz-tree
\newcommand{\SmokeyTreeIndent}{\leftmargini}
\newcommand{\SmokeyTreeSkip}{\baselineskip}
\makeatletter
\newcount\dirtree@lvl
\newcount\dirtree@plvl
\newcount\dirtree@clvl
\def\dirtree@growth{%
	\ifnum\tikznumberofcurrentchild=1\relax%
		\global\advance\dirtree@plvl by 1\relax%
		\expandafter\xdef\csname dirtree@p@\the\dirtree@plvl\endcsname{\the\dirtree@lvl}%
	\fi%
	\global\advance\dirtree@lvl by 1\relax%
	\dirtree@clvl=\dirtree@lvl%
	\advance\dirtree@clvl by -\csname dirtree@p@\the\dirtree@plvl\endcsname%
	\pgf@xa=\SmokeyTreeIndent\relax%
	\pgf@ya=-\SmokeyTreeSkip\relax%
	\pgf@ya=\dirtree@clvl\pgf@ya%
	\pgftransformshift{\pgfqpoint{\the\pgf@xa}{\the\pgf@ya}}%
	\ifnum\tikznumberofcurrentchild=\tikznumberofchildren%
		\global\advance\dirtree@plvl by -1\relax%
	\fi%
}
\tikzset{%
	dirtree/.style={%
		growth function=\dirtree@growth,%
		growth parent anchor=south west,%
		parent anchor=south west,%
		every node/.style={%
			%anchor=north,%
			% Need to set minimum height, inner/outer sep, and
			% text depth in order to make each node drawn the
			% same height regardless of ascenders and descenders
			% in the font
			minimum height=\baselineskip,%
			inner sep=0in,%
			outer ysep=0in,%
			outer xsep=1ex,%
			text depth=0.2ex%
		},%
		every child node/.style={%
			anchor=south west%
		},%
		edge from parent path={%
			([xshift=2ex] \tikzparentnode\tikzparentanchor)%
			|- (\tikzchildnode\tikzchildanchor)%
		}%
	}%
}
\makeatother

\newcommand{\SquareTextBullet}{\rule[0.3ex]{0.75ex}{0.75ex}}

% Documentation: http://mirrors.ibiblio.org/CTAN/macros/latex/contrib/enumitem/enumitem.pdf
\usepackage{enumitem}

% Diagram of the various dimensions for lists: http://www.ntg.nl/maps/11/33.pdf
%\setlist{parsep=0.1\baselineskip, itemsep=0.2\baselineskip}
\newlengthval{\StandardLabelSep}{0.5em}
\setlist{labelsep=\StandardLabelSep}

\newcommand{\ProcessLabelFormat}[1]{#1.}
\newlist{Process}{enumerate}{3}
\setlist[Process,1]{%
	label=\ProcessLabelFormat{\arabic*},%
	}
\setlist[Process,2]{label=\ProcessLabelFormat{\alph*}}
\setlist[Process,3]{label=\ProcessLabelFormat{\roman*}}

\newcommand{\ItemizeBulletFormat}[1]{#1}
\setlist[itemize,1]{label=\ItemizeBulletFormat{\textbullet}}
\setlist[itemize,2]{label=\ItemizeBulletFormat{--}}

\newlist{Definition}{description}{3}
\newcommand{\DefinitionBulletFormat}[1]{#1}
\newcommand{\DefinitionFormat}[1]{%
	\makebox[0in][r]{\DefinitionBulletFormat{\textbullet}\hspace{\StandardLabelSep}}%
	#1\textnormal{ --- }%
}
\setlist[Definition]{%
	leftmargin=\leftmargini,%
	labelsep=0in,%
	labelindent=\leftmargin,%
	style=standard,%
	format=\DefinitionFormat%
}

\newlist{Property}{description}{1}
\newcommand{\PropertyBulletFormat}[1]{{\color{SmokeyBlue}#1}}
\newcommand{\PropertyFormat}[1]{%
	\makebox[0in][r]{\PropertyBulletFormat{\SquareTextBullet}\hspace{\StandardLabelSep}}%
	#1\textnormal{ --- }%
}
\setlist[Property]{%
	leftmargin=0in,%
	labelsep=0in,%
	labelindent=\leftmargin,%
	style=standard,%
	format=\PropertyFormat%
}

\newcounter{LogSectioni}
\newcommand{\LogSectionBulletFormat}[1]{\textnormal{#1}}
\newcommand{\LogSectionFormat}[1]{%
	\stepcounter{LogSectioni}%
	\hspace{\leftmargini}%
	\makebox[0in][r]{\LogSectionBulletFormat{\theLogSectioni.}\hspace{\StandardLabelSep}}%
	#1\textnormal{ --- }%
}
\newlist{LogSection}{description}{3}
\setlist[LogSection,1]{%
	labelsep=0in,%
	style=standard,%
	format=\LogSectionFormat,%
	before=\setcounter{LogSectioni}{0}%
}

\newlengthval{\SmokeyListingAboveSkip}{1.5\parskip}
\newlengthval{\SmokeyListingBelowSkip}{-\medskipamount}
\addtolength{\SmokeyListingBelowSkip}{1.5\parskip}

\newlengthval{\SmokeyTableParskipTop}{0.75\parskip}
\newlengthval{\SmokeyTableParskipBottom}{\parskip}

\usepackage{mylistings/listings}
\lstset{%
	frame=none,%
	keepspaces=true,%
	basicstyle=\footnotesize\ttfamily,%
	tabsize=4,%
	linewidth=4in,%
	minipagewidth=\textwidth,%
	aboveskip=\SmokeyListingAboveSkip,%
	belowskip=\SmokeyListingBelowSkip,%
}
\lstdefinestyle{LuaCodeStyle}{%
	% Why does \color{SmokeyBlue} not work?
	%backgroundcolor=\color{SmokeyBlue},%
	frame=single,
	rulecolor=\color[RGB]{252,251,220},%
	backgroundcolor=\color[RGB]{252,251,220},%
	linewidth=4in,%
	xleftmargin=\leftmargini%
}
\lstdefinestyle{SmokeyApiStyle}{%
	linewidth=4in,%
	xleftmargin=\leftmargini%
}
\lstdefinestyle{SettingStyle}{%
	linewidth=4in,%
	xleftmargin=\leftmargini%
}
\lstdefinestyle{TerminalOutputStyle}{%
	linewidth=4in,%
	xleftmargin=\leftmargini%
}
\lstdefinestyle{LogExcerptStyle}{%
	frame=single,%
	rulecolor=\color{SmokeyDarkBlue},%
	linewidth=\textwidth%
}
\lstdefinestyle{AnnotedLogFileStyle}{%
	frame=single,%
	rulecolor=\color{SmokeyDarkBlue},%
	linewidth=\textwidth,%
	basicstyle=\scriptsize\ttfamily,%
	xleftmargin=78pt,%
	tabsize=8,%
	breaklines=true,%
	breakatwhitespace=true,%
	breakindent=0in,%
	breakautoindent=false,%
	prebreak=\raisebox{-1ex}[0in][0in]{$\hookleftarrow$},%
	%postbreak=\raisebox{1ex}[0in][0in]{$\hookrightarrow\;$},%
}
\lstdefinestyle{CommandLineStyle}{%
	linewidth=4in%
}

\lstnewenvironment{LuaCode}[1][]%
	{%
		\lstset{style=LuaCodeStyle,#1,minipage}%
	}%
	{}
\lstnewenvironment{LuaApi}[1][]%
	{%
		\lstset{%
			style=SmokeyApiStyle,%
			escapechar=~,%
			#1,%
			inject=\raisebox{-0.125ex}[0in][0in]{\makebox[0in][r]{\makebox[\leftmargini][c]{\Large\ComputerMouse}}},%
			xleftmargin=\leftmargini,%
		}%
	}%
	{}
\lstnewenvironment{Setting}%
	{%
		\lstset{style=SettingStyle,escapechar=~}%
	}%
	{}
\lstnewenvironment{TerminalOutput}%
	{%
		\lstset{%
			style=TerminalOutputStyle,%
			escapechar=~,%
			inject=\raisebox{-0.125ex}[0in][0in]{\makebox[0in][r]{\makebox[\leftmargini][c]{\Large\Printer}}},%
			xleftmargin=\leftmargini,%
		}%
	}%
	{}
\newlength{\ElideShift}
\lstnewenvironment{LogExcerpt}%
	{%
		\lstset{style=LogExcerptStyle,escapechar=~}%
		\newcommand{\elide}{%
			\makebox[0in][l]{%
				% Align past the listing frame (based on framesep length)
				\setlength{\ElideShift}{-5.2pt}%
				\addtolength{\ElideShift}{-2em}%
				\hspace{\ElideShift}%
				\begin{tikzpicture}[%
					every path/.style={%
						decoration={%
							complete sines,%
							segment length=2em,%
							amplitude=0.5ex%
						},%
						decorate%
					}%
				]%
					% Knock-out color
					\draw[color=white, yshift=0.475ex, line width=0.575ex] (-2em-0.5pt,0) -- (\linewidth+6.7pt+2em,0);%
					% Match frame color
					\draw[color=SmokeyDarkBlue, yshift=0.85ex] (0,0) -- (\linewidth+6.7pt,0);%
					\draw[color=SmokeyDarkBlue, yshift=0.15ex] (0,0) -- (\linewidth+6.7pt,0);%
				\end{tikzpicture}%
			}%
		}%
	}%
	{}
\lstnewenvironment{AnnotedLogFile}%
	{%
		\lstset{style=AnnotedLogFileStyle,escapechar=~}%
		% Custom \marginpar-like command
		\renewcommand{\mp}[1]{%
			\makebox[0in][r]{%
				\color{SmokeyDarkBlue}%
				\raisebox{0in}[0in][0in]{%
					\fbox{%
						\normalfont\bfseries\scriptsize##1%
					}%
				}%
				\rule[0.5ex]{1em}{\fboxrule}%
				\hspace{3pt}%
			}%
		}%
	}%
	{}
\lstnewenvironment{CommandLine}%
	{%
		\lstset{%
			style=CommandLineStyle,%
			escapechar=~,%
			inject=\raisebox{-0.125ex}[0in][0in]{\makebox[0in][r]{\makebox[\leftmargini][c]{\Large\Keyboard}}},%
			% Why does the linewidth seem to change when xleftmargin is set?
			linewidth=4.25in,%
			xleftmargin=\leftmargini%
		}%
	}%
	{}

\usepackage{colortbl}
\usepackage{tabu}
\usepackage{multirow}

\usepackage{array}
\renewcommand{\arraystretch}{1.5}

\newenvironment{PropertyTable}%
	{%
		\vspace{\SmokeyTableParskipTop}%
		\small%
		\begin{tabu}{| >{\columncolor{SmokeyLightBlue}}m{1.5in} m{1in} m{1in} |}%
			\rowfont{\color{white}\cellcolor{SmokeyBlue}}%
			\taburulecolor{SmokeyBlue}%
			\hline%
	}%
	{%
			\hline%
		\end{tabu}%
		\vspace{\SmokeyTableParskipBottom}%
	}

\newenvironment{ArgumentTable}%
	{%
		\vspace{\SmokeyTableParskipTop}%
		\small%
		\begin{tabu}{| >{\columncolor{SmokeyLightBlue}}m{1.2in} m{3in} |}%
			\rowfont{\color{white}\cellcolor{SmokeyBlue}}%
			\taburulecolor{SmokeyBlue}%
			\hline%
	}%
	{%
			\hline%
		\end{tabu}%
		\vspace{\SmokeyTableParskipBottom}%
	}

\newenvironment{PropertyTableContains}%
	{%
		\vspace{\SmokeyTableParskipTop}%
		\small%
		\begin{tabu}{| >{\columncolor{SmokeyLightBlue}}m{1.5in} m{0.8in} m{2in} m{0.6in} |}%
			\rowfont{\color{white}\cellcolor{SmokeyBlue}}%
			\taburulecolor{SmokeyBlue}%
			\hline%
	}%
	{%
			\hline%
		\end{tabu}%
		\vspace{\SmokeyTableParskipBottom}%
	}

\newenvironment{ValueTable}%
	{%
		\vspace{\SmokeyTableParskipTop}%
		\small%
		\begin{tabu}{| >{\columncolor{SmokeyLightBlue}}m{1.25in} m{1in} |}%
			\rowfont{\color{white}\cellcolor{SmokeyBlue}}%
			\taburulecolor{SmokeyBlue}%
			\hline%
	}%
	{%
			\hline%
		\end{tabu}%
		\vspace{\SmokeyTableParskipBottom}%
	}

\newenvironment{PdcaTable}%
	{%
		\vspace{\SmokeyTableParskipTop}%
		\small%
		\newcommand{\mc}[2]{\multicolumn{##1}{c|}{##2}}%
		\begin{tabu}{| >{\columncolor{SmokeyLightBlue}}m{1.5in} | >{\centering}m{1in} | >{\centering}m{1in} | >{\centering}m{1in} |}%
			\rowfont{\color{white}\cellcolor{SmokeyBlue}}%
			\taburulecolor{SmokeyBlue}%
			\everyrow{\hline}%
			\hline%
	}%
	{%
			\hline%
		\end{tabu}%
		\vspace{\SmokeyTableParskipBottom}%
	}

\newenvironment{StatusScreenTable}%
	{%
		\vspace{\SmokeyTableParskipTop}%
		\small%
		\begin{tabu}{| >{\centering}m{0.22\textwidth} >{\centering}m{0.22\textwidth} >{\centering}m{0.22\textwidth} >{\centering}m{0.22\textwidth} |}%
			\rowfont{\color{white}\cellcolor{SmokeyBlue}}%
			\taburulecolor{SmokeyBlue}%
			\hline%
	}%
	{%
			\hline%
		\end{tabu}%
		\vspace{\SmokeyTableParskipBottom}%
	}

\newenvironment{ApiTable}%
	{%
		\vspace{\SmokeyTableParskipTop}%
		\small%
		\begin{tabu}{| >{\columncolor{SmokeyLightBlue}}m{1in} m{0.5in} m{0.5in} m{2.75in} |}%
			\rowfont{\color{white}\cellcolor{SmokeyBlue}}%
			\taburulecolor{SmokeyBlue}%
			\hline%
	}%
	{%
			\hline%
		\end{tabu}%
		\vspace{\SmokeyTableParskipBottom}%
	}

\newenvironment{LuaStruct}%
	{%
		\vspace{\SmokeyTableParskipTop}%
		\small%
		\begin{tabu}{| >{\columncolor{SmokeyLightBlue}}m{1in} m{0.5in} m{3.5in} |}%
			\rowfont{\color{white}\cellcolor{SmokeyBlue}}%
			\taburulecolor{SmokeyBlue}%
			\hline%
	}%
	{%
			\hline%
		\end{tabu}%
		\vspace{\SmokeyTableParskipBottom}%
	}

\newenvironment{DirectoryTree}%
	{%
		% These commands are a wrapper around the TikZ notation in order
		% to automatically add decorations around certain node labels.
		%
		%     \node {foo} 
		%         child { node {bar}
		%             child { node {baz} }
		%             }
		%         child { node {boo}
		%             child { node {bah} }
		%             }
		%         ;
		%
		% There's a lot of \expandafter because we need to expand
		% *EVERYTHING* at the top level before handing parsing control
		% over to TikZ, which does not do LaTeX expansion.
		%
		% Also, it appears that nesting \def or \edef inside another
		% \def or \edef just doesn't work, so \Root has a different
		% implementation style than \File or \Directory.
		\newcommand{\DiskRoot}[2]{%
			\edef\RootLabel{##1}%
			\edef\RootChild{##2}%
			\expandafter\expandafter%
			\expandafter\node%
			\expandafter\expandafter%
			\expandafter{%
				\expandafter\expandafter%
				\expandafter\FloppySym%
				\expandafter\expandafter%
				\expandafter{%
				\expandafter\expandafter%
				\expandafter}%
				\expandafter\expandafter%
				\expandafter\ %
				\expandafter\RootLabel%
			\expandafter}%
			\RootChild%
			;%
		}%
		\newcommand{\Root}[2]{%
			\edef\RootLabel{##1}%
			\edef\RootChild{##2}%
			\expandafter\expandafter%
			\expandafter\node%
			\expandafter\expandafter%
			\expandafter{%
				\expandafter\expandafter%
				\expandafter\OpenFolderSym%
				\expandafter\expandafter%
				\expandafter{%
				\expandafter\expandafter%
				\expandafter}%
				\expandafter\expandafter%
				\expandafter\ %
				\expandafter\RootLabel%
			\expandafter}%
			\RootChild%
			;%
		}%
		\newcommand{\File}[1]{child{node{##1}}}
		\newcommand{\ClosedDirectory}[1]{%
			\expandafter child%
			\expandafter{%
				\expandafter node%
				\expandafter{%
					\expandafter\noexpand%
					\expandafter\ClosedFolderSym%
					\expandafter{%
					\expandafter}%
					\expandafter\ %
					##1%
				}%
			}%
		}%
		\newcommand{\Directory}[2]{%
			\expandafter child%
			\expandafter{%
				\expandafter node%
					\expandafter{%
						\expandafter\noexpand%
						\expandafter\OpenFolderSym%
						\expandafter{%
						\expandafter}%
						\expandafter\ %
						\expandafter##1%
					\expandafter}%
				##2%
			}%
		}%
		%\hspace{\leftmargini}
		\small%
		\begin{center}%
		\begin{tikzpicture}[dirtree]%
	}%
	{%
		\end{tikzpicture}%
		\end{center}%
	}

\newlength{\SideBySideTopSep}
\newcommand{\SideBySideDirectories}[2]{%
	% Emulate the {center} environment's list-style formatting
	\setlength{\SideBySideTopSep}{\topsep}%
	\addtolength{\SideBySideTopSep}{\parskip}%
	% The following \raisebox calls place all content below the baseline,
	% so compensate by pulling them up one line
	\addtolength{\SideBySideTopSep}{-\baselineskip}%
	\vspace{\SideBySideTopSep}%
	\hfill%
	\makebox[0in][c]{%
		\raisebox{-\height}{%
		\begin{minipage}{0.5\linewidth}%
		#1%
		\end{minipage}%
		}%
	}%
	\hfill%
	\makebox[0in][c]{%
		\raisebox{-\height}{%
		\begin{minipage}{0.5\linewidth}%
		#2%
		\end{minipage}%
		}%
	}%
	\hfill\null\newline%
	% More {center} emulation
	\vspace{\topsep}%
}

\usepackage{graphicx}
\usepackage{twoopt}

\newlengthval{\CenterImageVerticalMargin}{1ex plus 2ex minus 1ex}
\newcommandtwoopt{\CenterImage}[3][\linewidth][\CenterImageVerticalMargin]{%
	\begin{centering}%
		\vspace{#2}%
		\includegraphics[width=#1]{#3}%
		\vspace{#2}%
	\end{centering}%
}

\usepackage{ifthen}
\makeatletter
\newcommand{\iftt}[1]{\ifthenelse{\equal{\f@family}{\ttdefault}}{#1}{}}
\makeatother

\usepackage{xstring}
\usepackage{verbatim}

% Sequence properties
\newcommand{\prop}[1]{{\itshape#1}}
\newcommand{\propval}[1]{{\itshape#1}}
% Library functions
\newcommand{\func}[1]{\mbox{\itshape#1}}
% NVRAM variables, Control bits
\newcommand{\nvram}[1]{\mbox{\itshape#1}}
\newcommand{\nvarg}[1]{\mbox{``#1''}}
% Jargon (Smokey terminology)
\newcommand{\jargon}[1]{{\bfseries#1}}
% User-specified values (including command line arguments)
\newcommand{\param}[1]{{\iftt{\bfseries}\itshape#1}}
% Language keywords or in-line code
\newcommand{\keyword}[1]{\mbox{\itshape#1}}
% Command line programs and their prameters (excluding Mac OS X GUI apps)
\newcommand{\cmdline}[1]{%
	\mbox{%
		\itshape%
		\StrSubstitute%
			{#1}%
			{--}%
			{{-}{-}}%
	}%
}
% File names
\newcommand{\filename}[1]{``#1''}
\newcommand{\dirname}[1]{{\itshape#1}}

\newcommand{\optional}{\textit{(optional)}}

\hyphenation{Number-Of-Times-To-Run}
\hyphenation{Factory-Logs-Waiting-To-Be-Collected}

%%%%%%%%%%%%%%%%%%%%%%%%%%%%%%%%%%%%%%%%%%%%%%%%%%%%%%%%%%%%%%%%%%%%%%%%%%%%%%

\begin{document}

\newgeometry{top=\TopMargin, bottom=\BottomMargin, hmargin=\TransverseMargin}

%%%%%%%%%%%%%%%%%%%%%%%%%%%%%%%%%%%%%%%%%%%%%%%%%%%%%%%%%%%%%%%%%%%%%%%%%%%%%%
%%                DO NOT CONVERT TABS TO SPACES IN THIS FILE                %%
%%                DO NOT CONVERT TABS TO SPACES IN THIS FILE                %%
%%                DO NOT CONVERT TABS TO SPACES IN THIS FILE                %%
%%%%%%%%%%%%%%%%%%%%%%%%%%%%%%%%%%%%%%%%%%%%%%%%%%%%%%%%%%%%%%%%%%%%%%%%%%%%%%
%% Vim settings                                                             %%
%%    set noexpandtab                                                       %%
%%    set spell spelllang=en spellfile=SmokeyManual.en.utf-8.add            %%
%%%%%%%%%%%%%%%%%%%%%%%%%%%%%%%%%%%%%%%%%%%%%%%%%%%%%%%%%%%%%%%%%%%%%%%%%%%%%%

\begin{titlepage}

\setlength{\parindent}{0in}

{
	\fontsize{32}{32}\selectfont
	\textcolor{AppleGray}{\AppleLogo}
}
\\[0.25in]

{
	\fontsize{32}{32}\selectfont
	\textcolor{SmokeyDarkBlue}{\bfseries Lua Hardware Access} \\
	\textbf{User and Technical Manual}
}
\\[0.25in]

%\textbf{``Spirit'' Edition} \\
%\makebox[0in][r]{%
%	\raisebox{-3pt}[0in][0in]{%
%		%\rule{2pt}{84pt}%
%		\rule{2pt}{135pt}%
%		}%
%	\hspace{6pt}%
%	}%
April XX, 2014
\\[0.25in]

Preethi Srinivasan \\
System Software Engineering, Firmware \\
\texttt{preethi\_srinivasan@apple.com}

\vfill

\begin{center}
Apple Confidential --- Internal Use Only
\end{center}

\end{titlepage}


\clearpage
\fancyhfoffset[r]{0in}
\pagenumbering{gobble}

\setcounter{tocdepth}{3}
\tableofcontents

\clearpage
\fancyhfoffset[l]{\BodyIndentLeft}
\fancyhfoffset[r]{0in}
% TODO: Why does using \restoregeometry instead of \newgeometry cause the
% following pages to shift downards?
%\restoregeometry
\newgeometry{top=\TopMargin, bottom=\BottomMargin, left=\LeftMargin, right=\TransverseMargin}
\pagenumbering{arabic}

%%%%%%%%%%%%%%%%%%%%%%%%%%%%%%%%%%%%%%%%%%%%%%%%%%%%%%%%%%%%%%%%%%%%%%%%%%%%%%
%%                DO NOT CONVERT TABS TO SPACES IN THIS FILE                %%
%%                DO NOT CONVERT TABS TO SPACES IN THIS FILE                %%
%%                DO NOT CONVERT TABS TO SPACES IN THIS FILE                %%
%%%%%%%%%%%%%%%%%%%%%%%%%%%%%%%%%%%%%%%%%%%%%%%%%%%%%%%%%%%%%%%%%%%%%%%%%%%%%%
%% Vim settings                                                             %%
%%    set noexpandtab tw=0 indentexpr=                                      %%
%%    set spelllang=en spellfile=SmokeyManual.en.utf-8.add                  %%
%%    set spell                                                             %%
%%%%%%%%%%%%%%%%%%%%%%%%%%%%%%%%%%%%%%%%%%%%%%%%%%%%%%%%%%%%%%%%%%%%%%%%%%%%%%

%%%%%%%%%%%%%%%%%%%%%%%%%%%%%%%%%%%%%%%%%%%%%%%%%%%%%%%%%%%%%%%%%%%%%%%%%%%%%%
%% TODO:
%%   - Change "Command Line Arguments" to "Command Line Options"?
%%   - Create a cross-reference table for "Setting Up the Host and DUT"
%%%%%%%%%%%%%%%%%%%%%%%%%%%%%%%%%%%%%%%%%%%%%%%%%%%%%%%%%%%%%%%%%%%%%%%%%%%%%%

\section{Introduction}

\subsection{What is Lua Hardware Access?}

Blah

\subsection{Audience}

Blah

\subsection{Document Objective}

This document is generally intended to be an all-around reference, but written
with the intent of bringing LHA's core users up to speed about the
API.  

On the producer side, this manual intends to educate software engineers about
the functionality and intent of the system.  The goal is to exposit the design
and shed light on the considerations required to either maintain or enhance the
software.

On the consumer side, this manual presents the features available in the LHA
API.  Additionally, time will be spent to document requirements,
prerequisites, and expected behaviors.  The goal is to arm users with the
knowledge to bootstrap their own LHA code.

%%%%%%%%%%%%%%%%%%%%%%%%%%%%%%%%%%%%%%%%%%%%%%%%%%%%%%%%%%%%%%%%%%%%%%%%%%%%%%

% TODO: Share between Smokey and LHA manual this section

\section{Lua Fundamentals}

\subsection{Lua}

The Lua language is a product of the Tecgraf group at PUC-Rio (Pontifical
Catholic University of Rio de Janeiro) and intended to be a powerful, fast,
lightweight, embeddable scripting language.  It has dynamic data types,
built-in support for associative arrays, and offers automatic memory
management.  Scripts are internally translated into byte code that is executed
by an internal virtual machine.

The name ``Lua'' is Portuguese for ``moon''.  Scripts in the Lua language have
the extension \filename{.lua} and are typically encoded in ASCII.

Lua is the adopted language of the Smokey environment.  On top of the benefits
listed above, Lua was chosen because both the interpreter and its built-in
libraries are easy to port to the EFI environment without compromising their
feature sets.  Additionally, the language, code base, and community support are
both mature and stable.

All interactions between users (i.e., scripts) and the platform (e.g., lower
software stack, hardware drivers) will be in the form of Lua function calls,
either direct or via tables, and Lua  data types.  Smokey provides custom
custom APIs to make these interactions easier to manage.

Since it is not the intent of this document to teach Lua (see References
section), the following is only meant to help kick-start seasoned programmers.

\subsection{Data}

\subsubsection{Built-in Types}

Lua supports 6 built-in types that are useful here.

\begin{Definition}

\item[nil] Its only value is \keyword{nil}

\item[boolean] Its only values are \keyword{true} and \keyword{false}

\item[string] Doubly and singly quoted text such as \keyword{"foo"} and
\keyword{'bar'}

\item[number] Decimal integer (\keyword{123}), Hex (\keyword{0x123}), and
floating point (\keyword{10.5})

\item[table] Associative array between a key of any type and a value of any
type

\item[function] An optionally named chunk of executable code declared by the
\keyword{function} keyword

\end{Definition}

\subsubsection{Dynamism}

Lua variables are dynamically typed.  That is, they do not have a set type and
will become the type of the value being assigned.  Lua will attempt to ``do the
right thing'' and cast between strings, booleans, and numbers.

There are no mechanisms to restrict types, there is an operator to inspect
types.

\subsubsection{Scope}

Variables in Lua are defined the moment they are used and exist until the end
of the code chunk in which it was used.

Between nested code chunks, it is important to remember that Lua variables are
lexically scoped.  In the case that the same variable name is amongst different
nesting levels, the nested chunks can ensure that they are using their own
variable by applying the \keyword{local} keyword during variable declaration.

\subsubsection{Multiple Values}

Lua supports assigning and returning multiple values at a time through use of
the comma operator.  Any missing value in a tuple is automatically assigned the
value \keyword{nil}.

\begin{LuaCode}
a, b = 1, 2   -- a = 1 and b = 2
c, d = nil, 3 -- c = nil and d = 3
e, f = 4      -- e = 4 and f = nil
\end{LuaCode}

\begin{LuaCode}
function foo ()
	-- Return two values
	return 1, 2
end
\end{LuaCode}

\subsection{Comments}

Lua comments are akin to C++ comments.

\subsubsection{Single-line Comments}

Basic comments start with a double dash and run until the end of the line.

\begin{LuaCode}
-- This line is not code
\end{LuaCode}

\begin{LuaCode}
x = 3 -- Assign 3 to the variable 'x'
\end{LuaCode}

\subsubsection{Delimited Comments}

\begin{LuaCode}
--[[
This line is not code
Nor is this line
--]]
\end{LuaCode}

\begin{LuaCode}
y = --[[ These words are not code --]] 5
\end{LuaCode}

\subsection{Boolean Expressions}

\subsubsection{Relational Operators}

Lua has all the basic relational operators.

\begin{LuaCode}
a > b  -- Greater than
a < b  -- Less than
a >= b -- Greater/Equal
a >= b -- Less/Equal
a == b -- Equal
a ~= b -- Not equal
\end{LuaCode}

\subsubsection{Logical Operators}

Logical operators are spelled out.

\begin{LuaCode}
(a < b) or (c < d)  -- Disjunction
(a < b) and (b < c) -- Conjunction
not (a < b)         -- Negation
\end{LuaCode}

\subsubsection{Conditionals}

Any value that is neither \keyword{nil} nor \keyword{false} is considered to be
a boolean false.  This means that numbers, strings, functions, and tables are
considered true.  Counterintuitively to C/C++ programmers, this also means that
the number 0 is considered true.

\begin{LuaCode}
nil             -- False
false           -- False
true            -- True
0               -- True
1.5             -- True
"foo"           -- True
{ a = 5 }       -- True
function () end -- True
\end{LuaCode}

\subsubsection{Checking Existence}

As a best practice, any code checking whether a variable is set should
explicitly compare against \keyword{nil} rather than using the variable itself
as the expression.

\begin{LuaCode}
-- Good practice check for existence
if (x ~= nil) then
	foo()
end
\end{LuaCode}

\begin{LuaCode}
-- Bad practice check for existence
if (x) then
	foo()
end
\end{LuaCode}

\subsection{Tables}

Lua tables are associative hashes and are used through the language for both
storage and also for program organization.

\subsubsection{Storage}

Defining tables with implicit (i.e., numeric) keys.

\begin{LuaCode}
t = { 10, 20, 30 }
u = { "a", "b", "c" }
v = { 10, "z", false }
\end{LuaCode}

Defining tables with explicit key-value pairs.

\begin{LuaCode}
w = { [1]=10, [10]=20, [100]=30 }
z = { a=1, b=2, c=3 }
\end{LuaCode}

\subsubsection{Access}

Typical table access use familiar bracket notation.

\begin{LuaCode}
a = t[1]
b = t[a]
\end{LuaCode}

Keys that are strings have special syntactic sugar.

\begin{LuaCode}
-- These two assignments are the same
b = z["a"]
c = z.a
\end{LuaCode}

Walking through a table can be done generally with a \keyword{for} loop and the
\func{pairs} iterator.  Note that traversal order is not guaranteed.

\subsubsection{Arrays}

Lua does not have a built-in array type.  Rather, it depends on tables whose
keys are contiguously numbered integers.  All Lua arrays start at index 1.

\begin{LuaCode}
t = { 10, 20, 30 }
foo(t[1]) -- The argument value is 10
foo(t[2]) -- The argument value is 20
foo(t[3]) -- The argument value is 30
\end{LuaCode}

Walking through an array can be done with a \keyword{for} loop and the
\func{ipairs} iterator.  Note that traversal starts with index 1 and ends when
the next consecutive index is undefined.

\subsubsection{Classes}

Lua doesn't have a class system per se, but it is possible to emulate their
behavior using tables with keys that are strings and values that are functions.

Lua provides some syntactic sugar to make class methods easy to declare.

\begin{LuaCode}
-- Declare method f() inside table t
t = {
	m = 1,
	f = function (self, a)
		return self.m + a
	end
}
\end{LuaCode}

\begin{LuaCode}
-- Declare method f() inside table t
t = {
	m = 1
}
function t:f (a)
	return self.m + a
end
\end{LuaCode}

Calling a class method and passing in the ``self'' pointer is has syntactic
sugar, too.

\begin{LuaCode}
x = t.f(t, 1, 2) -- The hard way
\end{LuaCode}

\begin{LuaCode}
x = t:f(1, 2)    -- Lua's syntactic sugar
\end{LuaCode}

The syntax \keyword{t:f()} implicitly passes \keyword{t} as the first argument
to \keyword{f}.  Note that it is possible and easy to mistakenly call
\keyword{f} using the normal table access syntax of \keyword{t.f()} without any
complaint from Lua.   However, your code may misbehave because it will not have
the correct arguments!

\subsection{Language Constructs}

The Lua syntax will look familiar to any programmers of ALGOL-influenced
languages like C/C++, Pascal, Python, or Perl.   

\subsubsection{Functions}

\begin{LuaCode}
function foo (x, y, z)
	return x + y + z
end
a = foo(1, 2, 3)
\end{LuaCode}

\begin{LuaCode}
bar = function (x, y, z)
	return x + y + z
end
b = bar(1, 2, 3)
\end{LuaCode}

\subsubsection{if-then-else}

\begin{LuaCode}
if (a > 0) then
	x = 1
end
\end{LuaCode}

\begin{LuaCode}
if (a > 0) then
	x = 1
else
	x = 0
end
\end{LuaCode}

\begin{LuaCode}
if (a > 10) then
	x = 2
elseif (a > 0) then
	x = 1
else
	x = 0
end
\end{LuaCode}

\subsubsection{Loops: for-do}

\begin{LuaCode}
-- Iterate from 1 to 10
for x = 1, 10 do
	foo(x)
end
\end{LuaCode}

\begin{LuaCode}
-- Iterate on all table entries
for key, value in pairs(t) do
	t[key] = value + 1
end
\end{LuaCode}

\begin{LuaCode}
-- Iterate on all array entries
for i, value in ipairs(t) do
	t[i] = value + 1
end
\end{LuaCode}

\subsubsection{Loops: while-do}

\begin{LuaCode}
-- Print powers of 2 less than 1000
-- that aren't foo()
x = 1
while (x < 1000) do
	x = x * 2
	if (foo(x)) then
		break
	end
end
\end{LuaCode}

\subsubsection{Loops: repeat-until}

\begin{LuaCode}
-- Add bar() while less than 1000 and
-- not foo()
x = 0
repeat
	x = x + bar()
	if (foo(x)) then
		break
	end
until (x > 1000)
\end{LuaCode}

\subsection{Built-in Facilities}

\subsubsection{String Concatenation}

Strings are joined with the double-dot operator.

\begin{LuaCode}
x = "to" .. "get" .. "her"
\end{LuaCode}

\subsubsection{\func{string.format} Function}

Lua supports a limited subset of the \func{printf} semantics from C.

\begin{LuaCode}
x = 100
s = string.format("%dKhz", x)
\end{LuaCode}

\subsubsection{Bitwise Operations}

Bit manipulation is available in the form of library calls rather than being
syntactically  integrated into the language like in C.  Operations are limited
to 32-bit quantities.

\begin{LuaCode}
-- Assume: uint32_t x, a, b, n
--         int32_t k
x = bit32.band(a, b)    -- x = a & b
x = bit32.bor(a, b)     -- x = a | b
x = bit32.bxor(a, b)    -- x = a ^ b
x = bit32.bnot(a)       -- x = ~a
x = bit32.lshift(a, n)  -- x = a << n
x = bit32.rshift(a, n)  -- x = a >> n
x = bit32.arshift(k, n) -- x = k >> n
x = bit32.lrotate(a, n) -- x = (a<<n) & (a>>(32-n))
x = bit32.rrotate(a, n) -- x = (a>>n) & (a<<(32-n))
if (bit32.btest(a, b))  -- if (a & b)
\end{LuaCode}

More complicated extractions and injections of fields are available as well.

\begin{LuaCode}
-- m = ((1 << w) - 1)
-- x = (a >> n) & m
x = bit32.extract(a, n, w)
\end{LuaCode}

\begin{LuaCode}
-- m = ((1 << w) - 1)
-- x = (a & ~(m << n)) | ((b & m) << n)
x = bit32.replace(a, b, n, w)
\end{LuaCode}

\subsubsection{\func{tonumber} Function}

Lua will try to convert values to numbers implicitly in some scenarios, but it
is often useful to do this manually.

\begin{LuaCode}
x = tonumber("100")
\end{LuaCode}

\subsubsection{\func{tostring} Function}

Lua will try to convert values to strings implicitly, but it is sometimes
useful to do this manually.

\begin{LuaCode}
x = 100
s = tostring(x) .. "KHz"
\end{LuaCode}

\subsubsection{\func{type} Function}

The type of a variable can be inspected using the \func{type} function.  It
returns a string describing the type.

\begin{LuaCode}
if (type(x) == "nil") then
	foo(1)
elseif (type(x) == "boolean") then
	foo(2)
elseif (type(x) == "string") then
	foo(3)
elseif (type(x) == "number") then
	foo(4)
elseif (type(x) == "function") then
	foo(5)
elseif (type(x) == "table") then
	foo(6)
else
	foo(7)
end
\end{LuaCode}

\subsection{Exceptions}

\subsubsection{Raising Exceptions}

The current context of execution can be aborted by raising an exception.
Arbitrary data can be sent up as part of the exception; descriptive strings are
helpful if the upper-most code can expect and print them.

\begin{LuaCode}
function f (x)
	if (x > 100) then
		error("Too large")
	end
end
\end{LuaCode}

\subsubsection{Catching Exceptions}

Exceptions can be caught by using Lua's \func{pcall} function to call code that
is anticipated to raise exceptions.  Note that the parameter needs to be a
function name, not a function call.

\begin{LuaCode}
pcall(f, 10) -- Safely execute f(10)
\end{LuaCode}

Exception handling will not be covered in detail here because Smokey will be
doing most of this work.  Attempts to intervene with exception processing many
lead to unexpected behavior in Smokey.

\subsection{Tricks}

\subsubsection{Data Alternation in Boolean Expressions}

Lua's logical \keyword{and} and logical \keyword{or} operators do not return
their results as a boolean type.  Instead, they return the operand whose value
evaluates to true.

\begin{LuaCode}
x = true and 1   -- The value is 1
y = nil and true -- the value is nil
\end{LuaCode}

\begin{LuaCode}
a = 3 or true    -- The value is 3
b = nil or "foo" -- the value is "foo"
\end{LuaCode}

In the case of functions that take optional arguments, this makes it easy to
provide a default value if none is specified.

\begin{LuaCode}
function f (x)
	-- If x is specified, return x + 1
	-- If not, return 100
	return (x or 99) + 1
end
\end{LuaCode}

\subsubsection{Ternary-Like Operator}

Lua does not support the question mark operator in C and C++.  However, the
in-line ``if-then-else'' semantics are partially supported by Lua boolean
expressions.

\begin{LuaCode}
-- Similar to:
--	if (a) then
--		x = b
--	else
--		x = c
--	end
x = a and b or c
\end{LuaCode}

The caveat is that the ``then'' part of the expression must not be false
because that situation will cause Lua to use the ``else'' portion.

\begin{LuaCode}
-- WARNING!  Both are same as x = c
x = a and nil or c
x = a and false or c
\end{LuaCode}

The workaround is to ensure that the ``then'' portion is always true.  This can
be done by negating the conditional and swapping the ``then'' and ``else''
portions.  If neither portion can be guaranteed to be true, the full
``if-then-else'' construct must be used instead.

%%%%%%%%%%%%%%%%%%%%%%%%%%%%%%%%%%%%%%%%%%%%%%%%%%%%%%%%%%%%%%%%%%%%%%%%%%%%%%

\section{LHA APIs}

\subsection{Using LHA}

To use LHA from your Lua scripts, simply require the 'hw' module:

\begin{LuaCode}
require('hw')
\end{LuaCode}

All APIs are exported as sub-tables under the top-level 'hw' table.  The following sections cover the various APIs currently provided.  Not all APIs will be available on all platforms.  Every effort will be made to document which platforms each API is available on.

%%%%
%%%%
%%%% SMBus
%%%%
%%%%

\subsection{SMBus/I2C APIs}
\label{subsec:SMBus}

The SMBus API allows the user to perform raw i2c reads and writes on any platform-supported bus, to any device (whether present on the platform or not).  Such flexibility allows for speedy prototyping on existing platforms, by fly-wiring new components to existing i2c busses (for example).  Implementers of this API should take care not to limit the support to known devices on the bus.

Implementers of this API must also take care to appropriate lock the i2c bus in use, such that no other parallel use is possible while executing reads and writes via this API.

The SMBus API is currently only available on the ARM Embedded EFI Diagnostics platform.

\subsubsection{read\_block}
\begin{LuaApi}
hw.smbus.read_block(bus, device_address, register_address, 
	register_address_size, length)
\end{LuaApi}

\begin{ArgumentTable}
	Argument & Description \\
	\prop{bus} & I2C bus number \\
	\prop{device\_address} & 7-bit device address \\
	\prop{register\_address} & Device register address \\
	\prop{register\_address\_size} & 1: 8-bit register addresses, 2: 16-bit register addresses \\
	\prop{length} & Number of bytes to read (non-zero) \\
\end{ArgumentTable}

This API performs an SMBus read\_block compatible read.  Read\_block reads perform the following:

\begin{Definition}
\item[Start Condition] Standard I2C/SMBus start condition
\item[Device Address + R/!W] Sends 7-bit device address, along with appropriate read/not-write bit
\item[Register Address] Sends 8- or 16-bit register address
\item[Repeated Start] Standard repeated start condition
\item[Read] Pulses the clock until length-number of bytes are read, or until the device returns a NACK
\item[Stop Condition] Standard stop condition
\end{Definition}

This API must take care to appropriately lock the bus, so that no other I2C consumer tries to concurrently drive traffic while this API is executing.  Such bus locking is platform dependent (Mac OS vs. iOS vs. EFI Diags vs. Mac EFI).

\paragraph{Example}

\begin{LuaCode}
local result, data = pcall(hw.smbus.read_block, 
	Bus, Address, Register, RegisterWidth, Count)

if(result == false) then
	error("Could not read register "..tohex(Register))
end

return data
\end{LuaCode}

\subsubsection{write\_block}
\begin{LuaApi}
hw.smbus.write_block(bus, device_address, register_address, 
	data_table, register_address_size, data_count)
\end{LuaApi}

\begin{ArgumentTable}
	Argument & Description \\
	\prop{bus} & I2C bus number \\
	\prop{device\_address} & 7-bit device address \\
	\prop{register\_address} & Device register address \\
	\prop{register\_address\_size} & 1: 8-bit register addresses, 2: 16-bit register addresses \\
	\prop{data} & Table of bytes to read \\
\end{ArgumentTable}

This API performs an SMBus write\_block compatible read.  Write block is similar to read block, except that it performs a write, and clocks \emph{out} data, instead of clocking it in.

\paragraph{Example}

\begin{LuaCode}

local Val_table = {}
Val_table[1] = 0x02

local result, data = pcall(hw.smbus.write_block, 
	Bus, Address, Register, Val_table, RegisterWidth, 
	#Val_table)

if (result == false) then
	error("Could not write register "..tohex(register))
end

return true
\end{LuaCode}

\subsubsection{quick\_read}
\begin{LuaApi}
hw.smbus.quick_read(bus, device_address)
\end{LuaApi}

\begin{ArgumentTable}
	Argument & Description \\
	\prop{bus} & I2C bus number \\
	\prop{device\_address} & 7-bit device address \\
\end{ArgumentTable}

Returns 'true' if the device at the given 7-bit address responds with an ACK to seeing its address on the bus with the read-flag.

\subsubsection{quick\_write}
\begin{LuaApi}
hw.smbus.quick_write(bus, device_address)
\end{LuaApi}

\begin{ArgumentTable}
	Argument & Description \\
	\prop{bus} & I2C bus number \\
	\prop{device\_address} & 7-bit device address \\
\end{ArgumentTable}

Similar to quick\_read, but sends the device address with the write-flag.

\subsubsection{send\_bytes}
\begin{LuaApi}
hw.smbus.send_bytes(bus, device_address, data_table)
\end{LuaApi}

\begin{ArgumentTable}
	Argument & Description \\
	\prop{bus} & I2C bus number \\
	\prop{device\_address} & 7-bit device address \\
	\prop{data\_table} & Bytes to send, similar to write\_block \\
\end{ArgumentTable}

Sends the given bytes on the bus immediately following the device address (sent with the write flag).  This is useful, for example, when writing to EEPROMs that do not have registers that need to be selected.

\subsubsection{receive\_bytes}
\begin{LuaApi}
hw.smbus.receive_bytes(bus, device_address, length)
\end{LuaApi}

\begin{ArgumentTable}
	Argument & Description \\
	\prop{bus} & I2C bus number \\
	\prop{device\_address} & 7-bit device address \\
	\prop{length} & Number of bytes to clock-in \\
\end{ArgumentTable}

Similar to send\_bytes, this API receives bytes from devices like EEPROMs.

\subsubsection{read\_word}
\begin{LuaApi}
hw.smbus.read_word(bus, device_address, register_address)
\end{LuaApi}

\begin{ArgumentTable}
	Argument & Description \\
	\prop{bus} & I2C bus number \\
	\prop{device\_address} & 7-bit device address \\
	\prop{register\_address} & Register address \\
\end{ArgumentTable}

Reads 16-bits from the given register address.

\subsubsection{write\_word}
\begin{LuaApi}
hw.smbus.write_word(bus, device_address, register_address, data)
\end{LuaApi}

\begin{ArgumentTable}
	Argument & Description \\
	\prop{bus} & I2C bus number \\
	\prop{device\_address} & 7-bit device address \\
	\prop{register\_address} & Register address \\
	\prop{data} & 16-bit data to write into the given register \\
\end{ArgumentTable}

Writes 16-bits to the given register.

\subsubsection{read\_byte}
\begin{LuaApi}
hw.smbus.read_byte(bus, device_address, register_address)
\end{LuaApi}

\begin{ArgumentTable}
	Argument & Description \\
	\prop{bus} & I2C bus number \\
	\prop{device\_address} & 7-bit device address \\
	\prop{register\_address} & Register address \\
\end{ArgumentTable}

Reads 8-bits from the given register address.

\subsubsection{write\_byte}
\begin{LuaApi}
hw.smbus.write_byte(bus, device_address, register_address, data)
\end{LuaApi}

\begin{ArgumentTable}
	Argument & Description \\
	\prop{bus} & I2C bus number \\
	\prop{device\_address} & 7-bit device address \\
	\prop{register\_address} & Register address \\
	\prop{data} & 16-bit data to write into the given register \\
\end{ArgumentTable}

Writes 8-bits to the given register.

%%%%
%%%%
%%%% UART
%%%%
%%%%

\subsection{UART APIs}
\label{subsec:UART}

The UART API allows the user to perform raw UART reads and writes on any platform-supported UART bus.

Implementers of this API must take care to appropriately lock the UART bus in use, such that no other parallel use is possible while executing reads and writes via this API.

The UART API is currently only available on the ARM Embedded EFI Diagnostics platform.

\subsubsection{flush\_tx}
\begin{LuaApi}
hw.uart.flush_tx(uart_port)
\end{LuaApi}

\begin{ArgumentTable}
	Argument & Description \\
	\prop{uart\_port} & UART number \\
\end{ArgumentTable}

Reset the UART TX portion of the UART specified by "uart\_port".

\subsubsection{flush\_rx}
\begin{LuaApi}
hw.uart.flush_rx(uart_port)
\end{LuaApi}

\begin{ArgumentTable}
	Argument & Description \\
	\prop{uart\_port} & UART number \\
\end{ArgumentTable}

Reset the UART RX portion of the UART specified by "uart\_port".

\subsubsection{get\_control}
\begin{LuaApi}
hw.uart.get_control(uart_port)
\end{LuaApi}

\begin{ArgumentTable}
	Argument & Description \\
	\prop{uart\_port} & UART number \\
\end{ArgumentTable}

Get the control settings for the given UART and return it in a table.

\subsubsection{set\_control}
\begin{LuaApi}
hw.uart.set_control(uart_port, control_table)
\end{LuaApi}

\begin{ArgumentTable}
	Argument & Description \\
	\prop{uart\_port} & UART number \\
	\prop{control\_table} & Control settings\\
\end{ArgumentTable}

Set the control settings for the given UART according to the given table.

\subsubsection{send}
\begin{LuaApi}
hw.uart.send(uart_port, data, length)
\end{LuaApi}

\begin{ArgumentTable}
	Argument & Description \\
	\prop{uart\_port} & UART number \\
	\prop{data} & Table of bytes to send \\
	\prop{length} & Number of bytes to send \\
\end{ArgumentTable}

Send the given bytes over UART.  Length is a necessary parameter, because "data" can contain NULL bytes.

\subsubsection{receive}
\begin{LuaApi}
hw.uart.receive(uart_port, length)
\end{LuaApi}

\begin{ArgumentTable}
	Argument & Description \\
	\prop{uart\_port} & UART number \\
	\prop{length} & Number of bytes to send \\
\end{ArgumentTable}

Read length number of bytes from the given uart and return them in a table.

\subsubsection{set\_configuration}
\begin{LuaApi}
hw.uart.set_configuration(uart_port, control_parameters)
\end{LuaApi}
\begin{ArgumentTable}
	Argument & Description \\
	\prop{uart\_port} & UART number \\
	\prop{control\_parameters} & Table of control parameters \\
\end{ArgumentTable}

Configure the various attributes of the UART specified by "uart\_port" using the table "control\_parameters" (timeout,baud rate,receivedepth,paity,databits,stopbits). If nothing is passed in, the current value will be retained.

\subsubsection{get\_configuration}
\begin{LuaApi}
hw.uart.get_configuration(uart_port)
\end{LuaApi}
\begin{ArgumentTable}
	Argument & Description \\
	\prop{uart\_port} & UART number \\
\end{ArgumentTable}

Return the current UART configuration in table.


\subsubsection{get\_rxfifo\_size}
\begin{LuaApi}
hw.uart.get_rxfifo_size(uart_port)
\end{LuaApi}
\begin{ArgumentTable}
	Argument & Description \\
	\prop{uart\_port} & UART number \\
\end{ArgumentTable}

Get the number of bytes left to receive on the particular UART specified by "uart\_port".

\subsection{SPI APIs}
\label{subsec:SPI}

SPI Blah

\subsection{Memory APIs}
\label{subsec:Memory}

Memory Blah

\subsection{GPIO APIs}
\label{subsec:GPIO}

GPIO Blah

\subsection{Helper APIs}
\label{subsec:Helpers}

Helper Blah

\iffalse
Smokey's sequence property list is built on top of Apple's ASCII plist format.
A schema comprised of predefined keys and values is defined in order to
describe both Smokey's behavior as well as the sequence's behavior.

Smokey's schema is rooted at the top of the plist hierarchy.  Properties
generally fall into one of two categories.  \jargon{Control properties} are
stored in the root.  \jargon{Test item properties} start at the root, but use a
recursive definition.  The meaning of these properties and their legal values
are described in the following sections.

\CenterImage{SequenceSchema.png}

\subsubsection{Structure}

The structure of test ordering in Smokey centers around the concepts of actions
and test items.  An \jargon{action} is a piece of Lua code that Smokey can
invoke.  A \jargon{test item} is a node in the plist structure with properties
identifying how the sequencer treats an action.  In effect, test items are
consumers of actions, and the Smokey sequencer is a consumer of test items.
Note that a particular action is allowed to appear in more than one test item.

The design of the schema is intended to produce nearly flat plist files for the
common case where a sequence is a linear list of test items.  In more complex
scenarios, actions can be grouped and nested by layering test items.  This is
done by defining a test item property rather than an action property inside a
test item.

\subsubsection{Control Properties}

These properties define the basic parameters that Smokey uses when processing a
sequence.  These properties must be at the root of the plist.  All values are
required.

\begin{PropertyTable}
	Control Properties & Type & Required? \\
	\prop{SequenceName} & String & Yes \\
	\prop{SequenceVersion} & String & Yes \\
	\prop{SchemaFormat} & Number & Must be ``1'' \\
	\prop{BrickRequired} & String & Yes \\
	\prop{BehaviorOnFail} & String & Yes \\
\end{PropertyTable}

\begin{Property}

\item[SequenceName] Text string describing name the sequence.

\item[SequenceVersion] User-defined identifier to track changes and
revisions.\footnote{Due to software limitations, \prop{SequenceVersion} must be
a number stored as a text string in the plist.} For example, this could be a
sequential number, or a date format like YYYYMMDD, where the letters are
replaced with digits for the calendar year, month, and day.

\item[SchemaFormat] Currently must be set to ``1''.

\item[BrickRequired] Apple charger required to start or continue the sequence.

	\begin{Definition}

	\item[None] No charger required and no checks performed.

	\item[Any] Charger will be checked but any known external charger will
	be accepted.

	\item[500mA] Sufficiently powered USB hub.

	\item[1A] B1 or equivalent.

	\item[2.1A] B9 or equivalent.

	\item[2.4A] B45 or equivalent.

	\end{Definition}

\item[BehaviorOnFail] Controls how Smokey proceeds after encountering a
failure.

	\begin{Definition}

	\item[KeepGoing] Continue with the rest of the sequence.

	\item[StopAfterFailedAction] Abort the sequence after an action fails.

	\end{Definition}

\end{Property}

\subsubsection{Test Root Properties}

These properties define the top of the test order and therefore required to be
defined at the root of the plist.  These are a subset of the test item
properties.

\begin{PropertyTableContains}
	Test Root Properties & Type & Required? & Contains \\
	\prop{FailScript} & String & Optional & \\
	\prop{NumberOfTimesToRun} & Number & Yes & \\
	\prop{Tests} & Array of Dict. & Yes & Test Item \\
\end{PropertyTableContains}

\begin{Property}

\item[FailScript] User-defined Lua function to invoke when a failure is
detected.

\item[NumberOfTimesToRun] The \prop{Tests} property will be executed this many
iterations.

\item[Tests] An ordered array of test items.

\end{Property}

\subsubsection{Test Item Properties}

The meat of the Smokey schema are the test items.  They are the basic
operating unit of the sequencer.  Test items may be nested within each other.

\begin{PropertyTableContains}
	Test Item Properties & Type & Required? & Contains \\
	\prop{TestName} & String & Optional & \\
	\prop{FailScript} & String & Optional & \\
	\prop{NumberOfTimesToRun} & Number & Yes & \\
	\prop{BehaviorOnAction} & String & Optional & \\
	\prop{ActionToExecute} & String & Alternate for \prop{Tests} & \\
	\prop{Tests} & Array of Dict. & Alternate for \prop{ActionToExecute} & Test Item \\
\end{PropertyTableContains}

\begin{Property}

\item[TestName] Text string naming the test item.  The effective test name
defaults to the value of \prop{ActionToExecute} if omitted.

\item[FailScript] User-defined Lua function to invoke when a failure is
detected.

\item[NumberOfTimesToRun] The \prop{Tests} or \prop{ActionToExecute} property
will be executed this many iterations.

\item[BehaviorOnAction] Additional steps required before invoking
\prop{ActionToExecute}.

	\begin{Definition}

	\item[None] Do nothing.  This is the default if \prop{BehaviorOnAction}
	is not specified.

	\item[SaveState] Save the state of the sequence to file.  If the DUT is
	interrupted due to power loss, reboot, or otherwise, the Smokey will
	automatically continue once the DUT boots up.

	\end{Definition}

\item[ActionToExecute] User-defined Lua function to invoke.

\item[Tests] An ordered array of test items.

\end{Property}

\subsection{Sequence Flow}

When Smokey executes a sequence, it handles a variety of tasks related to
preparing the DUT, executing actions, logging results, and reporting progress.
It will be important to know how these activities are arranged in order to
understand how sequence actions fit in.

\subsubsection{Pre-Flight Phase}

Before running a sequence, Smokey does the following:

\begin{Process}

\item Process command line options.

\item Check for autostart conditions.

	\begin{Process}

	\item If the autostart option is set, check the \nvram{boot-args} NVRAM
	variable.

		\begin{Process}

		\item If the \nvarg{smokey} argument is set, merge in those options.

		\item If the \nvarg{smokey} argument is not set, quit.

		\end{Process}

	\item If the autostart option is not set, continue normally.

	\end{Process}

\item Load the sequence's Lua script and parse the property list file.

\item Apply remaining command line and NVRAM options.

\item Open log files.

\item Record the start of the sequence.

\item Initialize hardware and EFI software features implied by the sequence
properties.

\item If \prop{BrickRequired} is set, wait for an external charger to be
connected.  The wait time is indefinite.

\end{Process}

\subsubsection{Sequence Execution Phase}

\begin{Process}[resume]

\item Traverse the \prop{Tests} array at the root of the plist in order and
process each element.  This is a pre-order, depth-first traversal.

	\begin{itemize}

	\item Failure handling occurs during this phase.  Fatal failures will
	force Smokey to jump to the next phase.

	\item If the test item defines \prop{ActionToExecute}, perform that action.

		\begin{Process}

		\item Flush pending file output

		\item Invoke the named Lua function

		\item Record the result of calling the function

		\end{Process}

	\item If the test item defines \prop{Tests}, recurse into that array.

	\end{itemize}

\end{Process}

\subsubsection{Post-Flight Phase}

\begin{Process}[resume]

\item Write complete results.

\item Flush pending file output.

\item If autostarted, reboot.

\end{Process}

\subsubsection{Periodic Tasks}

While a sequence is running, Smokey only has control of the system between
sequence actions.  That time is used to handle various repeating management
tasks.

\begin{Definition}

\item[Charger Check] If \prop{BrickRequired} is set and a change in the
external charger state is detected, wait for the charger to be reconnected.
The wait time is indefinite.

\end{Definition}

\subsection{Sequence Processing}

\subsubsection{Test Item Naming}

For reporting purposes, Smokey requires a name for each test item in the
sequence that defines \prop{ActionToExecute}.  If the \prop{TestName} property
is set, it is used as the test name.  Otherwise, the value of
\prop{ActionToExecute} is the default.

Note that the sequence properties must be configured to avoid duplicate test
names.  For instance, if several test items have the same
\prop{ActionToExecute}, each item must have a different \prop{TestName}.
Smokey will preemptively abort a sequence if it detects a naming conflict.

Test items without \prop{ActionToExecute} produce no results or data, so they
do not require a test name.

\subsubsection{Node Numbering}

For tracking and internal purposes, Smokey represents the sequence as a tree
and automatically gives each node a number.  Like the sequence execution phase,
assignment is done with a pre-order, depth-first traversal.  This produces node
numbers that read like line numbers when the sequence is printed in outline
form.

The root of the plist file is always node one.

Node number assignment will change as the sequence changes.  There is no
support for specifying a node's number.

\subsubsection{Number of Test Item Iterations}

The number of iterations per test item is the mathematical product of
\prop{NumberOfTimesToRun} at the test item in question and all test items
immediately above it.  One result is recorded each time the sequencer traverses
into the test item.

For example, consider a sequence with \prop{NumberOfTimesToRun} at the plist
root set to 2 and a test item with \prop{NumberOfTimesToRun} set to 3 and
\prop{ActionToExecute} defined.  Smokey will process the test item 6 times in
total.  The sequencer will traverse into the test item twice.  The first
traversal will record results for iterations 1 through 3.  The second traversal
will record results for iterations 4 through 6.

\subsection{Pass/Fail Criteria of Actions}

Smokey takes into account various Lua code return paths when assessing the
result of an action.  Scripts can take advantage of this to silently stop the
sequence or return verbose failure information for later failure analysis.

\subsubsection{Function Return Value}

\begin{ValueTable}
	Return Value & Assessment \\
	nil & Pass \\
	true & Pass \\
	false & Fail \\
	All Others & Unsupported \\
\end{ValueTable}

Note that, in the Lua language, \keyword{nil} can be returned explicitly with a
\keyword{return} statement or implicitly by ending a code block without a
\keyword{return}.

\subsubsection{Exceptions}

Any exceptions not caught by actions themselves will be caught by the
sequencer.  Smokey considers this a failure and acts accordingly.

Note that the value of the data associated with an exception will be logged.
If an action chooses to throw an exception, it is recommended that a helpful
text string be sent.

\subsection{Failure Handling}

\subsubsection{Sequence Excursions}

Once an action's result has been assessed, Smokey decides whether to continue
following the test order.

\begin{Definition}

\item[Pass] Continue the sequence in the defined test order.

\item[Fail] Divert the test order based on the test item configuration.

\end{Definition}

The test item property \prop{BehaviorOnFail} defines Smokey's exact failure
handling.

\begin{Definition}

\item[KeepGoing] Ignore the failure.

\item[StopAfterFailedAction] Stop the sequence immediately.  Invoke the
\prop{FailScript} defined but the current test item and others as appropriate.

\end{Definition}

Note that \prop{BehaviorOnFail} will affect sequence traversal, but does not
play a part in assessing the result of an action.

\subsubsection{Failures During a Sequence}

Sequences can choose to handle failures---whether for clean-up purposes or to
manage internal state---by defining the \prop{FailScript} property.  Smokey
will invoke the named Lua function when the sequencer assesses an action
failure.

Because test items can be nested, more than one \prop{FailScript} may be
invoked.  The first one to be executed would be at the test item that failed.
The next would be the one in the test item immediately outside of that nested
test item.  And so on from the epicenter to the outermost \prop{FailScript} at
the plist root.  Any test items  along this path without a \prop{FailScript}
definition will be ignored.

The exact timing of \prop{FailScript} invocations depends on
\prop{BehaviorOnFail}.

\begin{Definition}

\item[StopAfterFailedAction] The first \prop{FailScript} will be invoked
immediately after failure assessment.  Invocations at outer nesting levels
follow immediately, ignoring the remaining iterations due to
\prop{NumberOfTimesToRun} at those test items.

\end{Definition}

\subsubsection{Failures During Periodic Tasks}

Any failure during a periodic task is considered fatal.  If \prop{FailScript}
is defined at the plist root, that Lua function shall be invoked.  Thereafter,
the execution phase is aborted.

\subsection{Test Results}

\subsubsection{Results Tracking}

During the course of a sequence, Smokey tracks results individually.  There is
a result recorded for each iteration of each test item.  Additionally, there is
an overall result based on the individual results. 

All results are initialized during the pre-flight phase to ``incomplete''.
Pass and fail are recorded as the sequence progresses.  The overall result is
recorded once the last test item is finished or the sequence is aborted due to
failure.

\subsubsection{Test Results Naming}

Names for results are automatically generated by combining the test item name
with the iteration number.\footnote{See the ``Individual Test Results'' table
in section \NumNameRef{subsec:TestResults}.}

\subsubsection{Test Data Naming}

Smokey can collect numeric data and limits from each test item.  The data's
name is specified by the sequence script, but Smokey will make the name unique
by including the name of the test item and the iteration number.\footnote{See
the ``Test Data'' table in section \NumNameRef{subsec:TestResults}.}

\subsubsection{Propagation of Results}

To maintain consistency, the overall sequence result is computed from the
individual test item results.  Therefore, any failure at the item level forces
the overall sequence result to fail.  Incomplete tests are treated as
failures.

\subsection{Sequence State Saving}

\subsubsection{Sequence Continuation}

A unique feature of Smokey compared to canonical Lua scripting is the built-in
ability to pause a sequence and continue at a later time.  This means that any
action can reboot the DUT, power cycle, or otherwise interrupt the system and
Smokey will know how to pick up from the last test item without losing results.

As mentioned in the test item properties description, this behavior is enabled
by setting \prop{BehaviorOnAction} to \propval{SaveState}.  It is effective
only during the test item in which the property is defined.  If the test item's
action does not interrupt the DUT, Smokey will automatically clear the
continuation point.

\subsubsection{Requirements}

The DUT must meet the requirements for autostart.  See the autostart section
for more details.

\subsubsection{Serialization}

When enabled, Smokey performs the following immediately before invoking
\prop{ActionToExecute}.

\begin{Definition}

\item[Set Continuation Point] Mark the sequence to continue at the test item
immediately following the one defining \prop{BehaviorOnAction}.

\item[Save Smokey State] Write all internal Smokey data to a temporary file.

	\begin{itemize}

	\item Save the state of the sequence traversal.

	\item Save all results the same file.

	\end{itemize}

\item[Configure Autostart] Set Smokey to run automatically at next boot.

	\begin{itemize}

	\item Save the current \nvram{boot-args} and \nvram{boot-command} NVRAM
	variables.\footnote{See the autostart description for how NVRAM
	variables are used by Smokey.}

	\item Add a bare \nvarg{smokey} argument to \nvram{boot-args}.

	\item Set \nvram{boot-command} to \nvarg{diags}.

	\end{itemize}

\end{Definition}

\subsubsection{Resurrection}

When Smokey is invoked from the command line, it checks for the existence of a
saved state before loading the user-specified sequence.  If one is found, the
state is automatically resurrected and that state's sequence will be continued.

\begin{Definition}

\item[Load State Data] Resurrect the data saved during serialization.

	\begin{itemize}

	\item Reload all internal Smokey variables.

	\item Reload all results.

	\end{itemize}

\item[Reset Autostart] Disable the configuration from serialization.

	\begin{itemize}

	\item Restore \nvram{boot-args} and \nvram{boot-command}.

	\end{itemize}

\item[Delete Continuation Point] Clear the state file so that it can be reused.

\end{Definition}

%%%%%%%%%%%%%%%%%%%%%%%%%%%%%%%%%%%%%%%%%%%%%%%%%%%%%%%%%%%%%%%%%%%%%%%%%%%%%%

\section{Design for Factory Use}

\subsection{Objectives for Factory Use}

Smokey's objective is to automate testing for manufacturing scenarios.
Primarily, this means enabling a DUT to run commands on its own and emit data
to factory processes.  Several design decisions were made and features
implemented to support these goals:

\begin{itemize}

\item Support for multiple, user-defined sequences allows use both on and off
the manufacturing line.

\item Sequences are maintained separately from EFI diags and therefore don't
require intensive validation (e.g., on all stations) before roll-out.

\item Sequences are stored on the filesystem to enable identical deployment on
a large number of test units.

\item Sequence execution is logged to a file for failure analysis.

\item Results are emitted in DCSD's format for PDCA.

\item Execution can be triggered via NVRAM.

\item Timestamps identify hangs, performance issues, and process excursions.

\end{itemize}

\subsection{Station Behavior}

In traditional station software, a station ID is used for reporting results and
log collection.  The name of the station is collaboratively chosen by the
station DRI, program managers, and the factory software team.  A numeric
identifier serves to codify the station's place within the factory process.

Smokey isn't meant to directly replace station software because it wasn't
designed to plug into the manufacturing infrastructure.  However, it can be
used to control those aspects of a station that affect the state of the DUT.

Station behavior is enabled by giving sequences specific, reserved names.
Presently, the sequence names below are reserved for specific factory station
activities.  Consult an EPM or TDL for the most up-to-date list for a specific
project.

\begin{itemize}

\item Wildfire (ID 0xB4)

\end{itemize}

The list of station-related behavior in Smokey follows:

\begin{itemize}

\item Control bits

\end{itemize}

Sequences that do not have any of the reserved names do not receive special
treatment.  Use of reserved sequence names for non-station-related activities
is highly discouraged.

\subsection{Control Bits}

Control bits are a process control mechanism used to track test coverage in the
factory.  Because coverage is directly related to final product quality,
control bits can only be manipulated with the proper clearance.  Smokey strives
to ensure the security of control bits by acting as a middleman between
sequences and the DUT.

For sequences associated with a station ID, the respective control bit is
updated as the sequence executes in order to record gross progress.  This
generally happens alongside the writing of results to the filesystem, but may
occur less frequently because the granularity of control bit states is coarse.

The control bit state can be used as a rough indicator of progress.

\begin{Definition}

\item[Untested] Sequence has not been invoked or Smokey aborted early.

\item[Incomplete] Smokey processed the sequence and has recorded its start.

\item[Pass \textnormal{or} Fail] Sequence is complete.

\end{Definition}

Writing to control bits can not be disabled.

\subsection{Log Collection}

\subsubsection{DUT Identifier}

Smokey identifies the device under test by the system serial number
\nvram{SrNm} in syscfg.  If that key is not defined, the MLB serial number
\nvram{MLB\#} is used instead.  It is considered a fatal error when neither key
is defined.

\subsubsection{File Output}

Smokey's file output is designed to be compatible with LogCollector, with
emphasis on being human readable.  Towards that goal, files are written in
ASCII format and Smokey's output is sent to several files, as documented
elsewhere.  All files in a sequence's directory will be collected and stored in
PDCA.

Two files will be interesting to those investigating issues with log
collection:

\begin{Definition}

\item[.FactoryLogsWaitingToBeCollected] Will be filled with zeros if
PDCA.plist has valid data.  Otherwise should not be empty.

\item[PDCA.plist] The main output file for LogCollector.  Smokey writes results
to this file in a PDCA-specific schema.  This file is zero-filled on a fresh
unit.  At the start of a sequence, Smokey writes this file with a default
result to indicate a crash, hang, or power loss; it is later overwritten with
the actual results of the sequence.

\end{Definition}

\subsection{Test Results and Data}
\label{subsec:TestResults}

Information reported to the PDCA system includes the results for all tests as
well as the data gathered by those tests.  These are all stored in plist format
using DCSD's PDCA schema.

Because of the schema, all output is represented as test results.  The PDCA
properties \prop{testname}, \prop{subtestname}, and
\prop{subsubtestname}\footnote{The PDCA database reports \prop{testname},
\prop{subtestname}, and \prop{subsubtestname} as one concatenated string.} are
used as identifiers for individual tests.  The \prop{result} and
\prop{failure\_message} properties, as well as others, encode sequence output.
Specific values are discussed below.

\begin{PdcaTable}
	Overall Sequence Result & Pass & Fail & Incomplete \\
	\prop{overallResult} & ``PASS'' & \mc{2}{``FAIL''} \\
\end{PdcaTable}

\begin{PdcaTable}
	Individual Test Results & Pass & Fail & Incomplete \\
	\prop{testname} & \mc{3}{Test item's \prop{TestName} or \prop{ActionToExecute} property} \\
	\prop{subtestname} & \mc{3}{``Iteration '' + iteration number} \\
	\prop{subsubtestname} & \mc{3}{Not used} \\
	\prop{result} & ``PASS'' & \mc{2}{``FAIL''} \\
	\prop{failure\_message} & Not used & Actual message & ``Incomplete'' \\
\end{PdcaTable}

\begin{PdcaTable}
	Test Data & No Limits & Within Limits & Exceed Limits \\
	\prop{testname} & \mc{3}{Test item's \prop{TestName} or \prop{ActionToExecute} property} \\
	\prop{subtestname} & \mc{3}{``Iteration '' + iteration number} \\
	\prop{subsubtestname} & \mc{3}{Sequence's data name} \\
	\prop{value} & \mc{3}{Sequence's data value} \\
	\prop{units} & \mc{3}{Sequence's data units (if provided)} \\
	\prop{lowerlimit} & Not used & \mc{2}{Data lower limit} \\
	\prop{upperlimit} & Not used & \mc{2}{Data upper limit} \\
	\prop{result} & \mc{2}{``PASS''} & ``FAIL'' \\
	\prop{failure\_message} & \mc{2}{Not used} & Automatic \\
\end{PdcaTable}

%%%%%%%%%%%%%%%%%%%%%%%%%%%%%%%%%%%%%%%%%%%%%%%%%%%%%%%%%%%%%%%%%%%%%%%%%%%%%%

\section{Using Smokey}

\subsection{Command Line Arguments}
\label{subsec:CmdArgs}

Smokey is command line driven and only supports certain combinations of
arguments.

In the following descriptions, options are prefixed by double dashes.  Options
shown in square brackets may be omitted.  Ellipses are used to indicate that
additional options may be specified, but omitted from this text for brevity.

Values to be supplied by the user are defined below:

\begin{Definition}

\item[Sequence] The name of a sequence.  Smokey searches for this folder name
in the standard location.  Must contain files appropriate for the DUT platform.

\end{Definition}

\subsubsection{Print Version}

\begin{CommandLine}
smokey --version
\end{CommandLine}

Show the build date and version.  There will be separate lines of output for
each component of the program.  Please include this information when filing
bugs or requesting help.

\subsubsection{Print Command Line Help}

\begin{CommandLine}
smokey --help
\end{CommandLine}

Print a list of all supported command line options.  Some options may not be
intended for general use.  Not all option combinations are valid.

\subsubsection{Run a Sequence}

\begin{CommandLine}
smokey ~\param{Sequence}~ [--clearstate] --run
\end{CommandLine}

Execute a sequence from a fresh state.

If there is an existing saved state, specifying \cmdline{--clearstate} ensures
that the saved state is not used.

\subsubsection{Continue from a Saved State}

\begin{CommandLine}
smokey [--retainstate] --run
\end{CommandLine}

If a saved state is available, continue from where it left off.  Most other
command line options are ignored.

Typically, the saved state is deleted when continuing a sequence so that
progress moves forward.  If the state should be retained to later re-run from
the same point, \cmdline{--retainstate} can be specified.  Note that continuing
twice from the same saved state may not produce the same file output as normal.
Also note that this option won't prevent a new state from overwriting the
current state.

\subsubsection{Sanity Check a Sequence}

\begin{CommandLine}
smokey ~\param{Sequence}~
\end{CommandLine}

Smokey will load the sequence script and plist files to check syntax and
settings.  No actions beyond that are taken.

\subsubsection{Perform a Dry Run on a Sequence}

\begin{CommandLine}
smokey ~\param{Sequence}~ [--clearstate] --dryrun
\end{CommandLine}

A dry run is the same as a normal run, but actions are not invoked.  This is
useful to make sure basic issues in Smokey are not causing sequence failures.

\subsubsection{Get Information about a Sequence}

\begin{CommandLine}
smokey ~\param{Sequence}~ --sequence
\end{CommandLine}

\begin{CommandLine}
smokey ~\param{Sequence}~ --summary
\end{CommandLine}

The first command dumps the entire sequence definition whereas the second
summarizes the actions that will be taken.  They can be used to verify the
structure of the sequence plist file.  See section \ref{subsec:SeqOut} for more
details.

\subsubsection{Clear Existing Saved State}

\begin{CommandLine}
smokey --clearstate
\end{CommandLine}

Erase the sequence state saved by \prop{BehaviorOnAction}.  No actions are
taken, and no errors reported, if there is no state presently defined.

\subsubsection{Autostart (Trigger Smokey Externally)}

\begin{CommandLine}
smokey ... --autostart
\end{CommandLine}

Decide whether or not to actually run Smokey based on external factors.  Quit
without error when those factors are not satisfied.  Follow autostart behavior
when all factors are present.

\subsection{Autostart}

On Darwin mobile devices, Smokey can be configured to execute a sequence upon
booting into EFI diags.  This allows automatic test execution on the device, as
well as integration with other test environments.  Smokey supports this feature
primarily to enable EFI-based testing from inside iOS.

\subsubsection{Prerequisites for Autostart}

Autostart is effected by a combination of device settings and EFI diags
configuration.  The following requirements must be met.

\begin{Definition}

\item[NVRAM] The DUT must support iOS-style NVRAM variables in EFI diags.  This
means that firmware driver must be available and the DUT must support this
hardware feature.

\item[EFI Diags Boot Configuration] On the EFI diags side, Smokey needs to be
part of the boot process.  This boils down to making the Smokey command part of
the built-in start-up script.  

\end{Definition}

\subsubsection{Configuring EFI Diags for Autostart}

It is recommended that the command line be executed as early as possible in the
boot process.

\begin{CommandLine}
smokey --run --autostart
\end{CommandLine}

\subsubsection{Configuring DUT for Autostart}

Once prerequisites are met, the following NVRAM variables can be set to kick
off a Smokey sequence.

\begin{Definition}

\item[boot-args] The \nvarg{smokey} argument must be set.  Other arguments in
this variable are ignored by Smokey.

\item[boot-command] Must be set to \nvarg{diags}.  If this is not possible, a
special cable or dongle will be required to force the DUT to boot into EFI
diags.

\item[auto-boot] Must be set to \nvarg{true}.

\end{Definition}

\subsubsection{Smokey Argument}

The \nvarg{smokey} argument in \nvram{boot-args} is a mechanism to augment the
Smokey command line.  Its value is a comma-separated list that is applied after
command line arguments have been processed.

The decision was made to put \nvarg{smokey} in \nvram{boot-args}---rather than
as its own NVRAM variable---in order to support deployment:  The
\nvram{boot-args} variable can be easily set from PurpleRestore, whereas other
variables are not well supported.  Units can have a sequence rooted and be
configured to run one on the next reboot.

Syntax and legal values are defined below.

\begin{Setting}
boot-args = "... smokey ..."
\end{Setting}

\begin{Setting}
boot-args = "... smokey=~\param{Sequence}~ ..."
\end{Setting}

There is currently only one NVRAM setting supported.

\begin{Definition}

\item[Sequence] The name of the sequence to execute.  If omitted, the sequence
named on the command line is used instead.

\end{Definition}

\subsubsection{Autostart Behavior}

As mentioned in the sequence flow description, the conditions for automatically
starting is the autostart option on the command line and the existence of
\nvarg{smokey} in \nvram{boot-args}.  If either are missing, Smokey will quit
without changing the state of the DUT.

When all autostart conditions are met, Smokey will proceed to run as if the
settings in the \nvarg{smokey} argument had been typed at the command line.
Additionally, the following tasks are performed:

\begin{itemize}

\item Remove the \nvarg{smokey} argument from \nvram{boot-args}.  This signals
external processes that Smokey has recognized and accepted the autostart
conditions.

\item Configure the DUT to run iOS on the next reboot.  This is done by setting
the \nvram{boot-command} NVRAM variable to \nvarg{fsboot}.

\end{itemize}

Automatic start implies automatic reboot.  Smokey will reboot the DUT at the
end of the sequence regardless of what transpires during its execution.

\subsection{Sequence Output}
\label{subsec:SeqOut}

\subsubsection{Run-time Output}

Smokey emits diagnostic and informative data as it runs.  The order is fixed
and the data logged to console and file are basically the same.  However, some
ways of invoking Smokey may omit some sections.

\begin{LogSection}

\item[Software Build Information] Source code version and binary build date.

\item[Device Identification] MLB and SoC identification numbers.

\item[Sequence Files] Files nominally used the current sequence.  Some files
may not be used, pursuant to sequent properties.

\item[Sequence Properties] Summary of the sequence properties in effect.

\item[Pre-flight Output] Running status as the DUT is prepared for the
sequence.

\item[Test Item Trace] A trace as Smokey traverses the sequence.  Each line is
prefixed with a context information.  EFI diags commands and direct script
output are interspersed in real time.

	\begin{Definition}

	\item[Timestamp] Shown in square brackets to separate from other text
	on the line.

	\item[Node Number] Smokey-assigned node number for the test item.
	Replaced with an ellipsis when it is the same as the line above.

	\end{Definition}

\item[Post-Flight Output] The sequence wind-down.  This includes the overall
sequence result as well as Smokey's attempt to write all results to file.

\item[Error Summary] The ``All Errors'' heading is used to reiterate all
failures and errors captured during a sequence.  This section is omitted when
the sequence passes.

\end{LogSection}

\subsubsection{Log File}

All output that Smokey emits to the console is also captured to the sequence's
log file.  Additionally, a time stamp is appended to the log each time it is
opened.  The following example shows the log file of a failed sequence,
including all of the output elements described previously.

\begin{AnnotedLogFile}
~\mp{Log Info}~Opened log at 2012-10-09 12:20:57

~\mp{1. SW Build}~Smokey 1A0529 (changelist 294521)
Built 2012/10/09 02:16:45

~\mp{2. Device ID}~SrNm: CCQHQ06QF4K0
MLB#: C02219500L9F4MQ1
CFG#: DA9/EVT2/00L9//2030/FT1-G2
ECID: 000001D7DC612DC9

~\mp{3. Seq. Files}~Control File: nandfs:\AppleInternal\Diags\Logs\Smokey\Wildfire\N78.plist
Script File:  nandfs:\AppleInternal\Diags\Logs\Smokey\Wildfire\N78.lua
Log File:     nandfs:\AppleInternal\Diags\Logs\Smokey\Wildfire\Smokey.log
Results File: nandfs:\AppleInternal\Diags\Logs\Smokey\Wildfire\PDCA.plist
Control Bit:  Wildfire (0xB4)

~\mp{Log Info}~Finished dumping pre-log buffer

~\mp{4. Seq. Props}~SequenceName:    N78 QT demo
SequenceVersion: 1
BehaviorOnFail:  StopAfterFailedAction
ResultsBehavior: Bookend
LogBehavior:     Full
BrickRequired:   1A

~\mp{5. Pre-flight}~Sequence syntax and sanity check passed

Writing default results
Writing control bit
Writing PDCA plist file

Initializing display
Initializing charger
Device ready

~\mp{6. Test Trace}~Sequence execution...

Day/Time      Node
------------- ----
[09 12:21:04] N001 Repeating 1x
[09 12:21:04] .... 	[1] Periodic tasks
[09 12:21:04] .... 		Detected 1A brick
[09 12:21:04] N002 	[1] Repeating 0x "Battery Protection"
[09 12:21:04] N003 	[1] Repeating 1x "PMU Test"
[09 12:21:04] .... 		[1] PmuTest
[09 12:21:04] .... 			pmureg -r 0 0x00
------------------------------------------------------------------------------
~\mp{Diags Command}~:-) pmureg -r 0 0x00
Register 0x0000 : 0x56
------------------------------------------------------------------------------
[09 12:21:04] .... 			Exit code = 0x00000000
[09 12:21:04] .... 			pmureg -r 0 0xA4
------------------------------------------------------------------------------
~\mp{Diags Command}~:-) pmureg -r 0 0xA4
Register 0x00A4 : 0xB5
------------------------------------------------------------------------------
[09 12:21:04] .... 			Exit code = 0x00000000
[09 12:21:04] .... 			pmustat chipid
------------------------------------------------------------------------------
~\mp{Diags Command}~:-) pmustat chipid
PMU Status test
ChipID: 0x56
------------------------------------------------------------------------------
[09 12:21:05] .... 			Exit code = 0x00000000
[09 12:21:05] N004 	[1] Repeating 2x "Gyro Test"
[09 12:21:05] .... 		[1] GyroTest
[09 12:21:05] .... 			gyro --init
------------------------------------------------------------------------------
~\mp{Diags Command}~:-) gyro --init
Powering on Gyro: OK
Resetting Gyro: OK
Gyro ChipID: ST Micro AP3GDL v3.0.0 part:3
Raw ID: 0xD5
Serial: 0x000000000
Warning: lot/batch numbers are not valid data at this time.
Lot: 0 Batch: 0
OK
------------------------------------------------------------------------------
[09 12:21:05] .... 			Exit code = 0x00000000
[09 12:21:05] .... 			gyro --selftest
------------------------------------------------------------------------------
~\mp{Diags Command}~:-) gyro --selftest
Starting Gyro self-test:
Self Test data:
(3)(-360)DATA: X-diff=363
(1)(362)DATA: Y-diff=361
(0)(-358)DATA: Z-diff=358
OK
------------------------------------------------------------------------------
[09 12:21:06] .... 			Exit code = 0x00000000
[09 12:21:06] .... 			gyro --off
------------------------------------------------------------------------------
~\mp{Diags Command}~:-) gyro --off
Powering down axes: OK
Powering down Gyro: OK
------------------------------------------------------------------------------
[09 12:21:07] .... 			Exit code = 0x00000000
[09 12:21:07] .... 		[2] GyroTest
[09 12:21:07] .... 			gyro --init
------------------------------------------------------------------------------
~\mp{Diags Command}~:-) gyro --init
Powering on Gyro: OK
Resetting Gyro: OK
Gyro ChipID: ST Micro AP3GDL v3.0.0 part:3
Raw ID: 0xD5
Serial: 0x000000000
Warning: lot/batch numbers are not valid data at this time.
Lot: 0 Batch: 0
OK
------------------------------------------------------------------------------
[09 12:21:07] .... 			Exit code = 0x00000000
[09 12:21:07] .... 			gyro --selftest
------------------------------------------------------------------------------
~\mp{Diags Command}~:-) gyro --selftest
Starting Gyro self-test:
Self Test data:
(4)(-360)DATA: X-diff=364
(1)(362)DATA: Y-diff=361
(0)(-358)DATA: Z-diff=358
OK
------------------------------------------------------------------------------
[09 12:21:09] .... 			Exit code = 0x00000000
[09 12:21:09] .... 			gyro --off
------------------------------------------------------------------------------
~\mp{Diags Command}~:-) gyro --off
Powering down axes: OK
Powering down Gyro: OK
------------------------------------------------------------------------------
[09 12:21:09] .... 			Exit code = 0x00000000
[09 12:21:09] N005 	[1] Repeating 1x "WiFi/BT Test"
[09 12:21:09] .... 		[1] WifiBtTest
[09 12:21:09] .... 			device -k WiFi -e power_on
------------------------------------------------------------------------------
~\mp{Diags Command}~:-) device -k WiFi -e power_on
ERROR: Method "power_on" returned status Not Found
device returned Not Found error
------------------------------------------------------------------------------
[09 12:21:09] .... 			Exit code = 0x8000000E
[09 12:21:09] .... 			ActionToExecute failed
[09 12:21:09] .... 		[1] WifiBtHandler
[09 12:21:09] N001 Sequence done

~\mp{7. Post-flight}~Failed

Writing final results
Writing control bit
Writing PDCA plist file

~\mp{8. Error Summary}~All errors:
	SmokeyResults: failed action WifiBtTest at node 5 iteration 1/1: EfiCommand: command had errors: device -k WiFi -e power_on
	SmokeyCore: stopping after failed action
\end{AnnotedLogFile}

\subsubsection{Sequence Dump}

\begin{LogExcerpt}
:-) smokey Wildfire --sequence
~\elide~
Test Sequence

Day/Time      Node
------------- ----
[09 11:13:17] N001 Repeat 1x
[09 11:13:17] N002      Repeat 0x "Battery Protection"
[09 11:13:17] ....              BatteryProtection
[09 11:13:17] N003      Repeat 1x "PMU Test"
[09 11:13:17] ....              PmuTest
[09 11:13:17] N004      Repeat 2x "Gyro Test"
[09 11:13:17] ....              GyroTest
[09 11:13:17] N005      Repeat 1x "WiFi/BT Test" --> WifiBtHandler
[09 11:13:17] ....              WifiBtTest
\end{LogExcerpt}

As a diagnostic and informative feature, Smokey can print the test order of a
sequence without doing anything else.  The \cmdline{--sequence} command line
option will print the test order very similar to the way that Smokey executes
it with \cmdline{--run}.

\begin{Property}

\item[Exhaustive Listing] All test items are included, even those with
\prop{NumberOfTimesToRun} set to zero.

\item[Explicit Properties] The properties \prop{TestName} and
\prop{NumberOfTimesToRun} are shown.  \prop{FailScript} is on same line as
\prop{NumberOfTimesToRun}, prefixed with an arrow.

\item[Actions] \prop{ActionToExecute} is shown on its own line.

\end{Property}

\subsubsection{Sequence Summary}

\begin{LogExcerpt}
:-) smokey Wildfire --summary
~\elide~
Test Sequence Summary

Day/Time      Node
------------- ----
[09 11:13:17] N001 Repeat 1x
[09 11:13:17] N003      PmuTest
[09 11:13:17] N004      Repeat 2x
[09 11:13:17] ....              GyroTest
[09 11:13:17] N005      WifiBtTest --> WifiBtHandler
\end{LogExcerpt}

The \cmdline{--summary} command line option is similar to the
\cmdline{--sequence} option, but potentially a lot more concise.

\begin{Property}

\item[Concise Listing] No test items with zero iterations.  Children of those
test items are also omitted.

\item[Abbreviated Properties] Test items with only one iteration do not have a
separate output line indicating showing \prop{NumberOfTimesToRun}.
\prop{TestName} is omitted on all test items.  \prop{FailScript} is on the same
line as \prop{ActionToExecute} if the line with \prop{NumberOfTimesToRun} is
omitted.

\end{Property}

\subsection{Screen Output}

If the DUT has a display module available, Smokey provides visual feedback on
the sequence's execution using modal screens.  Each has a distinct color or
design to make it easy to identify state at a glance.

\begin{StatusScreenTable}
	Connect to Brick &  Running Status & Pass & Fail \\
	\CenterImage{ScreenConnectToBrick.pdf} &
	\CenterImage{ScreenRunning.pdf} &
	\CenterImage{ScreenPass.pdf} &
	\CenterImage{ScreenFail.pdf} \\
\end{StatusScreenTable}

\begin{Property}

\item[Connect to Brick] Sequence is running, but halted while DUT is waiting
for an external charger.  Background is bright yellow and instructions in black
text.

\item[Running Status] DUT is running an action.  A logo is shown in the
background.  The current action name is shown in white text over a black
background.

\item[Pass] Sequence is complete and successful.  Background is bright green.

\item[Fail] Sequence is complete but failed.  Background is bright red.

\end{Property}

\subsection{State Control}

Smokey will generally manage its state without need for outside intervention.
However, commands are available to clear the state and get the DUT back to
normal.  These are useful, for example, for triaging a problem or interrupting
a sequence while the DUT is rebooting.

To understand why these commands work, please be sure to read the section on
sequence state saving before proceeding.

\subsubsection{Clearing State}

In order to run a new sequence, or restart a previous sequence, the previously
saved state must be cleared.  See the description of command line options for
the exact syntax to clear the current Smokey state.

Smokey only supports one state to be shared amongst all sequences.
Effectively, clearing one saved state will clear them for all sequences.

\subsubsection{Clearing Autostart}

When Smokey is part of the EFI diags boot script, the most important part of
breaking into a DUT in the middle of saving sequence state and rebooting is to
disable the autostart configuration.  This will usually allow EFI diags to boot
into the command line.

The work boils down to interrupting the DUT before it loads EFI diags in order
to manipulate NVRAM variables.

Breaking into the DUT before it loads EFI diags involves controlling it while
it is in either iBoot or iOS.  The details of doing this won't be covered here.

Once control is gained over the DUT, the following NVRAM variables can be
changed:

\begin{Definition}

\item[boot-args] The \nvarg{smokey} argument must be removed.

\item[boot-command] If the goal is not to boot into EFI diags, then this needs
to be set accordingly.  Otherwise, the value can be left untouched.

\item[auto-boot] Like \nvram{boot-command}, this can usually be left alone, but
can be changed depending on the situation.  Note that Smokey doesn't modify
this variable.

\end{Definition}

Additional work---beyond the scope of this document---may be required if other
software in EFI diags outside of Smokey configured for autostart at the same
time.

%%%%%%%%%%%%%%%%%%%%%%%%%%%%%%%%%%%%%%%%%%%%%%%%%%%%%%%%%%%%%%%%%%%%%%%%%%%%%%

\section{Using Smokey Simulator}

Smokey Simulator is a companion tool, built from the same source code as
Smokey, for executing and validating sequences from the command line.  Whereas
the \cmdline{smokey} command in EFI diags runs on a DUT, the simulator runs on
a host and provides a comparable level of functionality for the Mac OS
environment.  Additionally, it provides the capability to delegate EFI diags
commands to a linked DUT.

\subsection{Differences from Smokey}

Internally, the simulator is a port of Smokey, so they are largely identical.
This means the simulator can be used for validation and syntax checking of test
sequences, and, furthermore, to execute the sequences themselves.

Differences arise during interactions with the run-time environment.
Deviations occur because Smokey Simulator runs on a host and has the point of
view of a Mac OS application, whereas Smokey runs on a DUT and has the
perspective of an EFI diags application, so their views of the world are
different.

\begin{Definition}

\item[File Paths] Both Smokey and Smokey Simulator pass along file paths to the
run-time environment verbatim, so they are identical in this regard.  However,
because a DUT will require EFI-style paths and a host will require UNIX-style
paths, tests will need to either special-case the difference themselves or use
Smokey API functions like \func{FindSequenceFile}\footnote{See section
\NumNameRef{subsec:SmokeyApi}.\label{fn:DiffApi}} to factor out that logic.

\item[Input Files] Sequences and shared modules are treated as being in the
current working directory by default.  Relative and absolute path overrides are
supported for input files that are spread across the host filesystem.

\item[Output Files] File output is written to the input directories configured
at the start of the simulation. Contrast this to Smokey on a DUT, where input
locations are fixed, so output files are therefore at fixed locations as well.

\item[Ancillary Files] Smokey uses certain external files that are laid down
during a restore.  Smokey Simulator likewise requires these files to pre-exist
and also creates files of its own to emulate DUT processes.

	\begin{Definition}

	\item[State.txt] Prequisite.  File must exist, but can be empty.

	\item[ControlBit.txt] Output file.  Created at run time.

	\end{Definition}

\item[Control Bits] Since Mac OS does not have control bits in the same sense
as Apple mobile products, reads and writes are faked.  Control bits always read
\nvarg{Untested}.  Writes are recorded to the ancillary output file
\filename{ControlBit.txt} in text format.

\item[NVRAM] Non-volatile system variables are not supported.  Any Smokey
actions in the simulator that require the \nvarg{smokey} argument shall behave
as if it were not set.

\item[Platform Identification] Smokey Simulator promulgates the fictitious Z99
platform to test sequences.  Command line settings can be used to override this
identity piecemeal, as can link mode.  Lua code can call the
\func{PlatformInfo}\footref{fn:DiffApi} API function to detect whether the
sequence being executed by Smokey Simulator.

\item[Autostart] Not supported.

\item[External Commands] In the simulator, the Smokey API function \func{Shell}
ultimately calls the C library function \func{system}.  As a result, external
commands are routed through the host system's shell.  This behavior can be
used, to a limited degree, for targeted validation if the inputs and outputs of
EFI diags commands can be emulated by mock command scripts.  When live
execution of EFI commands is required, link mode should be used.

\end{Definition}

\subsection{Emulating the DUT Work Environment}

Smokey assumes fixed locations\footnote{See section
\NumNameRef{subsec:FileOrg}.} on the DUT for storing test sequences, modules,
and ancillary files.  Smokey Simulator does not make the same assumption and
must consider the layout of the host filesystem.  We can reconcile the two by
imagining that Smokey uses its home folder as the current working directory.
Thus, the sequences names in a \cmdline{smokey} command line can be seen as
relative paths to directories.  Substituting the \cmdline{smokey} command name
with \cmdline{SmokeySimulator} results in a simulator command line that mimics
its EFI counterpart.  This is the recommended way of calling with the simulator
because it minimizes command line arguments and makes switching between
environments easier.

Emulating the DUT work environment requires a directory configured with the
proper files and folders.  One way to get such a directory is to use the
\filename{Smokey} directory from a full EFI diags root or restore bundle.
Another way is to use the simulator release archive, which also includes
workflow helper files.  Lastly, a directory can be created from scratch, as
outlined below.  Once the directory is set up, use the shell \cmdline{cd}
command to change the working directory before invoking Smokey Simulator.

Manually crafting a directory tree for Smokey Simulator involves creating
ancillary files and setting up the corequisites of the test sequence of
interest.  There is currently only one required ancillary file,
\filename{State.txt}, so most effort will center around the auxiliary files and
modules of the sequence.

In the very minimal case, three files must be created, as shown below.
Advanced development environments may have multiple sequences in the working
directory, as well as a \filename{Shared} directory for shared modules.  In
both scenarios, it may be helpful to either copy or create a symbolic link to
the simulator binary.

\SideBySideDirectories
{
	\begin{DirectoryTree}
	\Root{Minimal Working Directory}{
		\File{State.txt}
		\Directory{Sequence}{
			\File{Main.lua}
			\File{Main.plist}
		}
	}
	\end{DirectoryTree}
}
{
	\begin{DirectoryTree}
	\Root{Working Directory}{
		\File{SmokeySimulator}
		\File{State.txt}
		\ClosedDirectory{SequenceOne}
		\ClosedDirectory{SequenceTwo}
		\ClosedDirectory{SequenceThree}
		\Directory{Shared}{
			\ClosedDirectory{ModuleA}
			\ClosedDirectory{ModuleB}
			\ClosedDirectory{ModuleC}
		}
	}
	\end{DirectoryTree}
}

Note that Smokey Simulator can still run if the above layout is not possible.
Refer to the command line reference for the arguments required to tell the
simulator where to find requisite files.

\subsection{Link Mode}

Link mode is a method of delegating the execution of EFI diags commands to a
tethered DUT.  Rather than merely simulating a test sequence via dry run or
mock command scripts, it is possible to execute a sequence on a host very close
to the way it would on a DUT.  Additionally, link mode shortens edit-run-debug
cycles by skirting the need to transfer files back and forth, and optimizing
the number of DUT reboots.

\subsubsection{System Requirements}

The simulator link is an RPC mechanism built upon a UART channel between host
and DUT.  Several pieces of hardware and software are required on both sides to
make it work.

\begin{itemize}

\item Host-side requirements

	\begin{itemize}

	\item Recent build of Smokey Simulator

	\item Bacon, Kong, or similar cable for UART communication

	\item \cmdline{usbterm} application \optional

	\end{itemize}

\item DUT-side requirements

	\begin{itemize}

	\item EFI diags with RPC-enabled version of Smokey

	\item iBoot-based firmware \optional

	\end{itemize}

\end{itemize}

All recent versions of EFI diags satisfy the RPC requirement.  If a manual
check is required, run the following command on the DUT:

\begin{CommandLine}
smokey --rpc
\end{CommandLine}

This will produce an \cmdline{RPC>} prompt if the diags image is compatible.
Otherwise, it will complain about unsupported command line arguments.  To exit
RPC mode and return to the EFI command shell, press \KeyCapsCtrlD.

Likewise, the simulator can be checked by either using the \cmdline{--link}
option or looking at the \cmdline{--help} output.

Both \cmdline{usbterm} and iBoot are required only when using Smokey Simulator
to boot the DUT into EFI diags.  The simulator option \cmdline{--diags} will
require both iBoot and \cmdline{usbterm}.  iBoot alone is sufficient when
booting the diags image already in the DUT's filesystem.

\subsubsection{Setting Up the Host and DUT}
\label{subsubsec:Setup}

Before the simulator link can be used, the DUT must be configured and the host
environment must be set up.

\begin{itemize}

\item Host-side configuration

	\begin{itemize}

	\item All other software using the UART link, including
	\cmdline{nanokdp}, must be terminated

	\end{itemize}

\item DUT-side configuration

	\begin{itemize}

	\item Disable \nvram{auto-boot} in iBoot \optional

	\item Root the desired EFI diags image onto the DUT \optional

	\item Run \cmdline{usbterm} in the background \optional

	\item DUT must be in either iBoot or EFI when starting the link

	\end{itemize}

\end{itemize}

All of the optional DUT settings center around boot behavior.  It is
recommended that \nvram{auto-boot} be disabled at the least.  However,
engineering efforts may require loading custom software, so Smokey Simulator
has allowances for getting the DUT into link mode.  The following list will
document the scenarios and settings involved, but won't delve into the nuances
of all the possible permutations of configuration settings.

\begin{itemize}

\item If the DUT will always be in EFI at the start of tethering and the test
sequence does not reboot the DUT, all the optional boot settings may be
omitted.  When reboots are involved, the remaining scenarios must be
considered.

\item If the EFI diags image on the DUT is sufficient, tethering can start
with the DUT in either diags or iBoot.  Reboots of the DUT during the test
sequence will require that \nvram{auto-boot} be disabled; alternatively,
\nvram{auto-boot} can be left enabled if \nvram{boot-command} is set to
\nvarg{diags}.

\item If a custom EFI diags is required and engineering efforts allow the DUT
to be modified, it sometimes saves time to root diags onto the device.  Doing
so will obviate \cmdline{usbterm}.  The DUT still needs a way to get into EFI
diags, so it must either be put into diags manually or left in iBoot for
Smokey Simulator to handle.  If reboots are expected, see the notes above
about \nvram{auto-boot} and \nvram{boot-command}.

\item If a custom EFI diags is required and the DUT's filesystem must be left
intact, Smokey Simulator can load the diags image dynamically.  This requires
\cmdline{usbterm} to be running in the background.  Smokey Simulator will need
access to iBoot, so the DUT must be in recovery mode at the start of
tethering.  If reboots are expected, \nvram{auto-boot} must be disabled so
that the simulator has a chance to re-load the diags image.

\end{itemize}

\subsubsection{Disabling Auto-Boot in iBoot}

The exact way to disable \nvram{auto-boot} will vary by environment.  To
disable from iOS: 

\begin{CommandLine}
nvram auto-boot=false
\end{CommandLine}

To disable \nvram{auto-boot} from iBoot: 

\begin{CommandLine}
setenv auto-boot false
saveenv
\end{CommandLine}

To disable \nvram{auto-boot} from EFI:

\begin{CommandLine}
nvram --set auto-boot false
nvram --save
\end{CommandLine}

\subsubsection{Choosing a Link Device}

Link mode requires the use of a communications channel between the host and DUT
for remote procedure calls.  The current implementation is built on top of UART
in order to have wide compatability with EFI, iBoot, and other environments.

Those using EFI diags are already familiar with the serial port on the DUT dock
connector.  This will typically be the same device used for the simulator link
because it is common to all environments on Darwin platforms.  In circumstances
where the DUT will not leave EFI diags, it is possible to use an alternate UART
such as USB CDC, which is much faster than standard serial devices.

Smokey Simulator needs the UNIX path to a character device representing the
UART connection to the DUT.  This is typically a file under the \dirname{/dev}
directory with the prefix \filename{cu.} followed by the device type and a
unique identifier.  The example below shows such a device provided by the Kong
hardware debugger.

\begin{Setting}
/dev/cu.kong-20063F
\end{Setting}

Alternatively, here is an example of a UART device exported by the EFI diags
USB CDC emulation driver.

\begin{Setting}
/dev/cu.usbmodemfd1411
\end{Setting}

Users of the \cmdline{nanokdp} command line option \cmdline{-d} can use the
same argument for the Smokey Simulator \cmdline{--link} option.  Other users of
\cmdline{nanokdp} can find the character device of an open session in the
escape menu.

\subsubsection{Using Link Mode}

TBD

DUT must be in either iBoot or EFI diags. (covered in Setting Up the Host and DUT)

how to execute DUT commands

smokey features that are delegated to dut over rpc tunnel

\begin{comment}

Smokey Simulator takes all the same arguments that the "smokey" EFI command
supports.  In addition, it accepts three that are unique to the simulation
environment.  All three are technically optional.

	SmokeySimulator
		/path/to/sequence/directory
		--link /path/to/uart
		--shared /path/to/Blaze/Smokey/Shared
		--diags /path/to/diags/img3/or/im4p


--link tells the simulator to connect to the target at the host file system’s device node
--shared allows direct use of platform-agnostic code on Blaze
--diags tells the simulator to boot the specific diags image when the target device is not in EFI

The --link option invokes the new "link" feature.  Its argument is a serial
device on the host file system, the same as the -d argument to nanokdp.  For
Kong, the serial interface has a path like /dev/cu.kong-20063F.  The nanokdp
utility can be used to test whether the path you want to use is valid:

	nanokdp -d /dev/cu.kong-20063F

Specifying --link causes Smokey Simulator to open the UART and issue commands
to query the state of the the DUT and bring the DUT into EFI.  It also tells
the simulator to create an RPC tunnel in order to hand off device-specific
commands to the DUT.

The --shared argument is optional.  The default is a directory named "Shared"
relative to the current working directory.  Each --shared argument adds to the
default.

Note that --diags only affects the transition from iBoot into EFI.  If the DUT
is already in EFI, the specified image is not loaded.  If this is an issue,
make sure to manually reset into iBoot beforehand.  Also note that --diags
requires usbterm to be running in the background.

%%

\end{comment}

\subsubsection{Quitting a Link}

Smokey Simulator will automatically terminate link mode and shut down the RPC
tunnel when it exits cleanly.  This will typically leave the DUT at the EFI
command line.

The simulator will also terminate link mode if it detects a reboot or shutdown.
The DUT's ultimate state will depend on the commands being executed at the
time.

If the simulator gets hung for some reason, it can be interrupted with
\KeyCapsCtrlC{}.  However, doing so can leave the DUT in the middle of the RPC
tunnel.  To clean up the DUT, connect to it using \cmdline{nanokdp} and press
\KeyCapsCtrlD{} to quit.  It may be necessary to press \KeyCapsEnter{} a few
times to get its attention before doing this.

\subsubsection{Continuing a Link}

Smokey Simulator will quietly and immediately exit if it detects that the DUT
has shutdown or rebooted.  This mimics the behavior of the DUT as seen with the
\cmdline{smokey} command.  When this happens, and \prop{BehaviorOnAction} was
set to \propval{SaveState}, Smokey Simulator can continue running the sequence
where it left off, with the sequence and simulator settings that were in effect
at that time, including link mode, without reiterating the previous command
line options.  To do this, simply use the \cmdline{--run} argument by itself.

\begin{CommandLine}
SmokeySimulator --run
\end{CommandLine}

If this is not desired, use the following command---by itself---to make Smokey
Simulator forget the previously saved state.

\begin{CommandLine}
SmokeySimulator --clearstate
\end{CommandLine}

\subsubsection{Loading an EFI Diags Image}

In link mode, Smokey Simulator will automatically boot EFI diags if it detects
that the DUT is in recovery mode.  By default, the diags image on the DUT will
be used, but a custom image can be specified at the command line\footnote{See
\NumNameRef{subsec:SimCmdArgs}.} for advanced development and debug scenarios.

See \NameRef{subsubsec:Setup} for configuration requirements.  Generallly,
\nvram{auto-boot} should be disabled and the DUT should be at the iBoot prompt.

Additionally, the \cmdline{usbterm} program is required in the background.  If
more than one device is connected to the host, the command line option
\cmdline{-locationid} can be used to specify the DUT that Smokey Simulator will
be controlling.

Once the link is up and running, the start-up banner will display the version
number of the EFI diags image on the DUT, underneath the simulator copyright
and version lines.  For example:

\begin{TerminalOutput}
Smokey Simulator - Copyright (C) 2012-2014 Apple Inc
Simulator 1A0035 (changelist 343409) 2014/02/24 15:10:09
N51 GreatDane 13C0586ae (changelist 42055) 2013/08/03 18:24:18
\end{TerminalOutput}

The above shows a DUT based on the N51 platform running EFI diags version
13C0586ae, as well as the date and time of the build.

\subsection{Command Line Arguments}
\label{subsec:SimCmdArgs}

Smokey Simulator builds upon the command line arguments of Smokey.  In addition
to the conventions and usage described in section \ref{subsec:CmdArgs}, several
simulator-specific arguments are available.

In the descriptions that follow, \cmdline{SmokeySimulator} should be
substituted with the correct path to the simulator binary.

\subsubsection{Specifying a Sequence}

\begin{CommandLine}
SmokeySimulator ~\param{Directory}~
\end{CommandLine}

Sequence names are treated as directory paths.  \param{Directory} can be a bare
name---unadorned with path separators---if it is in the current directory.
Relative and absolute paths are also supported.  Trailing slashes are tolerated
in order to accommodate shell tab completion.

\subsubsection{Specifying a Shared Directory}

\begin{CommandLine}
SmokeySimulator ... --shared ~\param{SharedDirectory}~
\end{CommandLine}

Smokey searches for shared code in a predefined directory.  In the simulator,
the default path is \filename{Shared}, which is presumed to be in the current
directory, but can be overridden from the command line.  The first use of
\cmdline{--shared} replaces the built-in default with \param{SharedDirectory}.
Subsequent uses append additional paths to the search list.

\subsubsection{Booting an EFI Diags Image}

\begin{CommandLine}
SmokeySimulator ... --diags ~\param{ImagePath}~
\end{CommandLine}

When link mode is enabled, load a specific EFI diags image if the DUT is in
recovery mode (i.e., at the iBoot prompt).  If the DUT is already in diags, the
currently running version will be used instead.

The file \param{ImagePath} must be either \filename{.img3} or \filename{.img4}
format, as appropriate for the DUT.

While the image is loading, \cmdline{usbterm} must be running in the background
to facilitate the transfer.  Additionally, Apple Connect may be required for
security reasons.

\subsubsection{Override Platform Identity}

\begin{CommandLine}
SmokeySimulator ... --platform ~\param{PlatformName}~
\end{CommandLine}

\begin{CommandLine}
SmokeySimulator ... --model ~\param{ModelName}~
\end{CommandLine}

\begin{CommandLine}
SmokeySimulator ... --boardid ~\param{IdNum}~
\end{CommandLine}

\begin{CommandLine}
SmokeySimulator ... --boardrevision ~\param{RevNum}~
\end{CommandLine}

Override the simulator's and the test sequence's default view of the platform
under test.  These options affect several functional areas.

Smokey uses the platform name when finding sequence and module files.  The
option \cmdline{--platform} influences this behavior and can be used to alter
the normal processing of a test sequence.

The \func{PlatformInfo}\footref{fn:DiffApi} API function will relay the
\param{PlatformName}, \param{ModelName}, \param{IdNum}, and \param{RevNum}
values to test code.

\subsubsection{Enabling DUT Link}

\begin{CommandLine}
SmokeySimulator ... --link ~\param{UartPath}~
\end{CommandLine}

Enable link mode.  The argument \param{UartPath} must point to a UART-like
character device in the file system.

No other program shall use the device at \param{UartPath} concurrently with
Smokey Simulator.  The most common cause for failure to enable link mode is a
background task using \param{UartPath} while the simulator accesses it.

This option overrides all of the platform identity options.

\subsubsection{Logging DUT Link}

\begin{CommandLine}
SmokeySimulator ... --rxlog ~\param{LogPath}~
\end{CommandLine}

When link mode is active, save raw data received from the DUT to
\param{LogPath}.  This option has no effect outside of link mode.  

When a new link is established, the file \param{LogPath} is overwritten.  Data
is appended to \param{LogPath} when Smokey Simulator reuses a link from a
previous session, such as when it continues from a saved state.

\subsection{Distribution}

The simulator is distributed via SDK and standalone archive.  The SDK release
makes the latest Smokey Simulator build readily available to developers who
have access to daily iOS builds.  Standalone releases are done regularly and
include example files.

Simulator releases are built independently of each other, so feature sets are
not in lock step, but they will generally have parity.  Neither are explicitly
synchronized with builds of Smokey for EFI diags.

\subsubsection{Smokey Simulator SDK Binary}

The Smokey Simulator binary is distributed as part of the iOS UI and NonUI
portion of the Apple-internal SDKs.  When Xcode command line tools are
available, the simulator binary can be invoked by name, as shown below.  Any
arguments past the simulator command name are passed directly to the
simulator.\footnote{Refer to section \NumNameRef{subsec:CmdArgs} for command
line tool conventions.}

\begin{CommandLine}
xcrun --sdk iphoneos SmokeySimulator [...]
\end{CommandLine}

When the overhead of calling \cmdline{xcrun} must be minimized, it is possible
to print the absolute path to \cmdline{SmokeySimulator} so that it may be
stored in a shell variable, alias, or symbolic link.

\begin{CommandLine}
xcrun --sdk iphoneos -f SmokeySimulator
\end{CommandLine}

\subsubsection{Release Archive}

The Smokey Simulator archive is distributed as a \filename{.zip} file on the
Smokey wiki.\footnote{See section \NumNameRef{subsec:AppleLinks}.}  In
addition to the simulator binary, the archive also includes ancillary files, a
\filename{Makefile}, sample sequences, and a directory for shared modules.  A
sample directory listing is shown below.

\begin{DirectoryTree}
\DiskRoot{SmokeySimulator.zip}{
	\File{Makefile}
	\File{SmokeySimulator}
	\File{State.txt}
	\ClosedDirectory{ComplexModuleDemo}
%	\Directory{ComplexModuleDemo}{
%		\File{Main.lua}
%		\File{Main.plist}
%	}
	\Directory{HelloWorld}{
		\File{.FactoryLogsWaitingToBeCollected}
		\File{Main.lua}
		\File{Main.plist}
		\File{PDCA.plist}
		\File{Smokey.log}
	}
	\ClosedDirectory{StIves}
%	\Directory{StIves}{
%		\File{.FactoryLogsWaitingToBeCollected}
%		\File{Main.lua}
%		\File{Main.plist}
%		\File{PDCA.plist}
%		\File{Smokey.log}
%	}
	\Directory{Shared}{
		\Directory{ComplexPlatformModule}{
			\Directory{Latest}{
				\ClosedDirectory{Platform}
%				\Directory{Platform}{
%					\File{N51.lua}
%				}
				\File{Common.lua}
				\File{Default.lua}
				\File{Package.lua}
				\File{PlatformSpecific.lua}
				\File{init.lua}
			}
		}
	}
}
\end{DirectoryTree}

The included \filename{Makefile} has rules for basic interaction with Smokey
files.

\begin{CommandLine}
make files
\end{CommandLine}

The \cmdline{files} target will regenerate a blank \filename{State.txt} file,
or create one if it does not exist.  The
\filename{.FactoryLogsWaitingToBeCollected}, \filename{PDCA.plist}, and
\filename{Smokey.log} files for the bundled sequences will be overwritten with
fresh copies.

\begin{CommandLine}
make clean
\end{CommandLine}

The \cmdline{clean} target will erase any ancillary output files that the
simulator creates during its execution.

\subsection{Working with Source Code}

TBD

\begin{comment}

The simulator link feature can make it easy to work with filesystem roots.  One
example of this is the Smokey directory in Blaze:  the recent reorganization
closely mimics the layout on the DUT.  To use Blaze with the simulator, one of
the following is required:

- Sequence is platform agnostic and files are stored under
	Blaze/Smokey/SEQUENCE/Main.plist
	Blaze/Smokey/SEQUENCE/Main.lua
- Sequence is platform-specific and files are stored under
	Blaze/Smokey/SEQUENCE/PLATFORM/Main.plist
	Blaze/Smokey/SEQUENCE/PLATFORM/Main.lua
- Sequence is platform-specific and files are stored under
	Blaze/Smokey/SEQUENCE/PLATFORM.plist
	Blaze/Smokey/SEQUENCE/PLATFORM.lua

where SEQUENCE is the specific sequence name (e.g. Wildfire) and PLATFORM is
the platform name according to EFI (e.g. N78, K93, N41).  For platform-specific
sequences, the simulator will detect the DUT platform and choose the
corresponding files.

Blaze is intended only for source code, but Smokey expects output files (e.g.
Smokey.log and PDCA.plist) to exist beforehand.  To aid in generating these
files, a Makefile has been checked into Blaze as Blaze/Smokey/Makefile.  To use
it, go to the Blaze/Smokey directory and run the command

	make files

Conversely, to remove the output files, run the command

	make clean

Using the simulator will require the binary be accessible from the Blaze/Smokey
directory.  This can be done either by copying the binary into Blaze/Smokey or
using the full path when calling the binary.

Once the files are set up, simulator usage is fairly straightforward:

1. "cd" into the Smokey directory
	cd /path/to/Blaze/Smokey
2. Run Smokey
	SmokeySimulator --link /path/to/uart [other arguments here]

Note that, at this point, the sequence name given at the command line will
actually be a directory name relative to the current working directory.  This
is a convenient way of emulating the command line used on the DUT.  Also note
that the default search path for shared Smokey files maps to
Blaze/Smokey/Shared under these circumstances, so the --shared argument is not
required.

To re-run a sequence that has already generated output, you can use "make
files" zero the output files.

\end{comment}

%%%%%%%%%%%%%%%%%%%%%%%%%%%%%%%%%%%%%%%%%%%%%%%%%%%%%%%%%%%%%%%%%%%%%%%%%%%%%%

\section{Developing Smokey Sequences}

\subsection{Smokey Lua API}
\label{subsec:SmokeyApi}

Smokey provides Lua scripting interfaces to interact with the DUT, EFI diags,
and with Smokey itself.  On top of the standard Lua facilities, these will be
the crux of sequence actions.  The following sections will describe the
Smokey API.

Some function arguments will be optional and will be noted with an asterisk.
Following the Lua convention, optional arguments at the end of a function's
argument list may simply be omitted during a call.  Optional arguments in the
middle must be passed as \keyword{nil}.

\subsubsection{\func{Shell} Function}

\begin{SmokeyApi}
Shell(~\param{CommandLine}~)
\end{SmokeyApi}

Execute the string \param{CommandLine} as if it was typed at the EFI command
shell.  Output is captured into the global table \func{Last}.

\begin{ApiTable}
	Argument & Type & Req? & Comment \\
	\param{CommandLine} & String & Yes & EFI diags command line \\
\end{ApiTable}

\func{Shell} will inspect the command output for failures.  If any are
detected, it will raise an exception with a string describing the first fault
detected.  If the sequence action doesn't catch this exception, Smokey will
catch it by default.

Failures are defined below.

\begin{Definition}

\item[Non-zero exit code] The convention for EFI commands is to return zero for
success.  All other values indicate some kind of failure.

\item[Error Message Detected]  The text ``ERROR:'' is the label for error
messages.  If this string is found, the command is considered to have failed
regardless of exit code.

\end{Definition}

\subsubsection{\func{Last} Table}

This global table is updated each time \func{Shell} is called.  Scripts can use
this table to inspect the most recent command's result or parse its output.

\begin{LuaStruct}
	\func{Last} Field & Type & Value \\
	\func{CommandLine} & String & The string passed into \func{Shell}. \\
	\func{ExitCode} & Number & Command return code. Zero means success. \\
	\func{RawOutput} & String & All console output from the command. \\
	\func{Output} & String & Same as \func{RawOutput} but stripped of error messages. \\
\end{LuaStruct}

\subsubsection{\func{ReportData} Function}

\begin{SmokeyApi}[linewidth=5in]
ReportData(~\param{Name}~, ~\param{Value}~, ~\param{Units*}~, ~\param{LowerLimit*}~, ~\param{UpperLimit*}~)
\end{SmokeyApi}

Record a named key-value pair for the current iteration of the current test
item.  Smokey will create a unique identifier for the data when generating PDCA
results.

\begin{ApiTable}
	Argument & Type & Req? & Comment \\
	\param{Name} & String & Yes & Local identifier for this datum. \\
	\param{Value} & Number & Yes & Value for this datum. \\
	\param{Units} & String & No & Dimension and magnitude (e.g., ``mV'' or ``s''). \\
	\param{LowerLimit} & Number & No & Must be in the same units as \param{Value}. \\
	\param{UpperLimit} & Number & No & Must be in the same units as \param{Value}. \\
\end{ApiTable}

\param{LowerLimit} and \param{UpperLimit} for \param{Value} will be
independently defined in the PDCA results if one or the other is provided in
the during the function call.

\param{Units} can be defined without defining limits.

\subsubsection{\func{ReportAttribute} Function}

\begin{SmokeyApi}
ReportAttribute(~\param{Name}~, ~\param{Value}~)
\end{SmokeyApi}

Record a key-value pair for the DUT.  PDCA treats this data as specific to the
unit rather than specific to any specific factory test.

\begin{ApiTable}
	Argument & Type & Req? & Comment \\
	\param{Name} & String & Yes & Global identifier for this attribute. \\
	\param{Value} & String & Yes & Value for this attribute. \\
\end{ApiTable}

There is a single namespace for attributes shared across the entire sequence.

\subsubsection{\func{FindSequenceFile} Function}

TBD

\subsubsection{\func{PlatformInfo} Function}

TBD

\subsection{Data Submission to PDCA}

Smokey defines \func{ReportData} as the main interface for generating
parametric data for the PDCA system.  Each datum is uniquely identified for
completeness.  See the section on design for factory use for more details on
the exact conversion process.

\subsection{Sequence Development Quick Start}

The quickest way to start developing a Smokey sequence is to duplicate an
existing one.  A member of the EFI diags team can provide these files.

However, it's possible to start from scratch by taking note of the file
organization described in the Smokey fundamentals section.  For example, the
following commands will create a sequence named ``MySequence'' for the X99
platform.  These commands need to be run at the command line of a host computer
while in the Smokey folder of a file tree root.

\begin{CommandLine}
mkdir MySequence
touch MySequence/X99.lua
touch MySequence/X99.plist
dd if=/dev/zero of=MySequence/PDCA.plist bs=1M count=1
dd if=/dev/zero of=MySequence/Smokey.log bs=1M count=10
echo SKIP > MySequence/.FactoryLogsWaitingToBeCollected
dd if=/dev/zero \
	of=MySequence/.FactoryLogsWaitingToBeCollected \
	count=1020 bs=1 seek=4
\end{CommandLine}

The Lua file will be created, but empty.  A text editor can be used to fill it
with the appropriate content.

The plist file will likewise be empty.  A text editor can also be used to fill
it, but a purpose-built application like Xcode is highly recommended.  At a
minimum, the required properties and one test item must be defined.

\CenterImage{Plist.png}

Once the files are created, they should be rooted onto the
DUT.\footnote{Consult iOS documentation for details on file roots, how they
work, and how to apply them to a DUT.}

\subsection{Developing Code with the Smokey Shell}

Certain features of Smokey are available as an interactive shell running on the
DUT.  This means that it is possible to test code snippets and EFI commands
without loading files onto the DUT.

Currently, this feature is limited to engineering hardware only.  Also, it must
be enabled in the particular build of EFI diags.

\subsubsection{Starting the Shell}

\begin{CommandLine}
smokeyshell
\end{CommandLine}

There are no command line options supported.

\subsubsection{Executing Commands}

The > (greater-than) prompt will be shown once the shell is running and ready
to accept input.  At this point, it will be possible to type or paste code into
the terminal window.

The shell supports all of the standard Lua functions and libraries, as well as
the standard interactive Lua shell conventions.  Additionally, the following
Smokey API functions may be used:

\begin{Definition}

\item[Shell] Behaves the same as in Smokey.

\end{Definition}

\subsubsection{Exiting the Shell}

Pressing \KeyCapsCtrlD{} will end input and quit Smokey Shell.

\subsection{Developing Code on DUT}

This feature is not yet available.

%%%%%%%%%%%%%%%%%%%%%%%%%%%%%%%%%%%%%%%%%%%%%%%%%%%%%%%%%%%%%%%%%%%%%%%%%%%%%%

\section{Smokey Internals}

\subsection{Software Architecture}

Documentation not yet available.

\subsection{Control Files}

Documentation not yet available.

\subsection{Saving State}

Documentation not yet available.

\subsection{Schema Grammar}

Documentation not yet available.

%%%%%%%%%%%%%%%%%%%%%%%%%%%%%%%%%%%%%%%%%%%%%%%%%%%%%%%%%%%%%%%%%%%%%%%%%%%%%%

\fi

\section{References}

\subsection{Useful Apple Links}
\label{subsec:AppleLinks}

\subsubsection{Smokey Announcements Mailing List}

smokey-announce@group.apple.com\footnote{Restricted access\label{fn:Restricted}}

\subsubsection{Smokey Wiki}

https://ipodwiki.apple.com/wiki/Smokey\footref{fn:Restricted}

\subsection{Useful Lua Links}

\subsubsection{Official Lua Home Page}

http://www.lua.org/home.html

\subsubsection{Lua Executable Binaries}

http://luabinaries.sourceforge.net/download.html

\subsubsection{Official Lua 5.2 Reference Manual}

http://www.lua.org/manual/5.2/

\subsubsection{Lua Programmer's Guide}

http://www.lua.org/pil/

\subsubsection{Lua Pitfalls and Gotchas}

http://www.luafaq.org/gotchas.html

\subsubsection{Lua Community Wiki}

http://lua-users.org/wiki/


\end{document}

% Replace the contents of this file with your text

\section{Introduction}

\subsection{What is this about?}

The new debug framework for EFI makes on-site or remote debugging a one or two
command effort. This is an extensive framework which covers a wide range of
platforms.

\subsection{Audience}

This document is intended for the engineers involved with the development and 
use of the EFI diagnostic environment.

\subsection{Document Objective}

This document will provide relevant technical details and concepts introduced 
in the new framework. It will also provide command usage guidelines 
that are useful to most users working in EFI.

\section{Setting up LLDB}

\subsection{Clone Repository}

Clone the diagssupport repository where all the scripts are located. \\

\cmdline{git clone git@gitlab.sd.apple.com:blackops/diagssupport.git}

\subsection{Create \filename{.lldbinit} file}

Its a good idea to create a \filename{\textasciitilde/.lldbinit} file so that you dont have to load the
scripts each time you start LLDB. You need to have the following information
inside the \filename{\textasciitilde/.lldbinit} file to use the framework effectively:

\cmdline{command script import <path-to-file>/efi.py}

\textbf{Note:} Here the <path-to-file> means the path where the files are 
located on your local machine

This file internally loads all the other scripts which you require. This means that 
now the user does not need to worry about which other files to include! The only 
caveat here is that the user needs to make sure that all the required script files
are in the same directory as the \filename{efi.py} file. 

\section{Using Framework}

\subsection{Available Commands}

The commands available in the framework are: \\

\textcolor{SmokeyDarkBlue}{\bfseries console}\\
\textcolor{SmokeyDarkBlue}{\bfseries ramlog}\\
\textcolor{SmokeyDarkBlue}{\bfseries symbols}\\
\textcolor{SmokeyDarkBlue}{\bfseries triage}\\
\textcolor{SmokeyDarkBlue}{\bfseries workspace}\\
\textcolor{SmokeyDarkBlue}{\bfseries loadpei}\\
\textcolor{SmokeyDarkBlue}{\bfseries hob}\\
\textcolor{SmokeyDarkBlue}{\bfseries build\_info}\\
\textcolor{SmokeyDarkBlue}{\bfseries config\_table} \\

\subsection{Console}

The console command gives you access to the forensics buffer within the device.

\textcolor{SmokeyDarkBlue}{\bfseries Usage:} \cmdline{console [--no-colors]}

\textcolor{SmokeyDarkBlue}{\bfseries Sample Output:} \\
\begin{TerminalOutput}
(lldb) console
BufferSize: 0x40000
Freespace: 0x3253b
Head: 0x0
Tail: 0xdac5
Data:0x87e2fe018
[Buf_PrintInfo]<0.0s (+0.000000)>Console router buffer allocated @ 0x87E2FE018, size = 262144
 [<no-func>]<14.656600s (+0.000006)>SetPS : Id 0x20E020168, State : 0x2
 [<no-func>]<14.656605s (+0.000005)>SetPS : Id 0x20E020170, State : 0x2
 [<no-func>]<14.656610s (+0.000005)>SetPS : Id 0x20E020178, State : 0x2
 [<no-func>]<14.656782s (+0.000172)>Install Display BackEnd done
 [<no-func>]<14.656934s (+0.000152)>Searching 5 DP Handles...
 [<no-func>]<14.656937s (+0.000003)>  T0x01 S0x03
 [<no-func>]<14.656941s (+0.000004)>  T0x01 S0x04
 [<no-func>]<14.656944s (+0.000003)>    G: 39AF9652-7356-4DA2-BB-C0-18-E8-05-00-99-4B
 [<no-func>]<14.656950s (+0.000006)>      DT:0x19
 [<no-func>]<14.656953s (+0.000003)>  T0x01 S0x04
 [<no-func>]<14.656955s (+0.000002)>    G: 9685762D-1A08-4C1E-A7-86-B9-62-9E-86-DF-69
 [<no-func>]<14.656960s (+0.000005)>  T0x01 S0x04
 [<no-func>]<14.656963s (+0.000003)>    G: 39AF9652-7356-4DA2-BB-C0-18-E8-05-00-99-4B
 [<no-func>]<14.656967s (+0.000004)>      DT:0x00
 [<no-func>]<14.657087s (+0.000120)>Searching 10 DP Handles...
 [<no-func>]<14.657091s (+0.000004)>  T0x01 S0x03
 [<no-func>]<14.657094s (+0.000003)>  T0x01 S0x04
 [<no-func>]<14.657097s (+0.000003)>    G: 39AF9652-7356-4DA2-BB-C0-18-E8-05-00-99-4B
 [<no-func>]<14.657101s (+0.000004)>      DT:0x19
 [<no-func>]<14.657104s (+0.000003)>  T0x01 S0x04
\end{TerminalOutput}

\subsection{Ramlog}

Similarly to the ~\param{command}~, the ~\param{ramlog}~ command gives you access to the
ramlog buffer if enabled on the device. This command could take several mintues to complete
depending on the size of the ramlog buffer.

\textcolor{SmokeyDarkBlue}{\bfseries Usage:} \cmdline{ramlog [--no-colors]}

\textcolor{SmokeyDarkBlue}{\bfseries Sample Output:} \\
\begin{TerminalOutput}
(lldb) ramlog
RamlogBuffer Struct Location: 0x9ea65c84
BufferSize: 0x2000000
BufferSizeMask: 0x1ffffff
MaxDataSize: 0x1000000
Head: 0x0
Tail: 0x397c
Data:0x9bfe9010
Reading 0x397c bytes from 0x9bfe9010
[_ReadGPADCInternal]<243.123884s (+0.008000)> ADC raw value = 2743
[_ReadGPADCInternal]<243.126094s (+0.002210)> ADC raw value = 2744
[_ReadGPADCInternal]<243.128301s (+0.002207)> ADC raw value = 2746
[_ReadGPADCInternal]<243.130512s (+0.002211)> ADC raw value = 2747
[ApplyCalibrationNew]<243.130564s (+0.000052)> This channel has no assocoated cal group
[measureMainADCChannels]<243.132446s (+0.001882)> vbus: unsupported
[_ReadGPADCInternal]<243.136925s (+0.004479)> ADC raw value = 2743
[_ReadGPADCInternal]<243.139134s (+0.002209)> ADC raw value = 2744
[_ReadGPADCInternal]<243.141349s (+0.002215)> ADC raw value = 2743
[_ReadGPADCInternal]<243.143562s (+0.002213)> ADC raw value = 2741
[ApplyCalibrationNew]<243.143612s (+0.000050)> This channel has no assocoated cal group
[measureMainADCChannels]<243.145252s (+0.001640)> acc_id: unsupported
[measureMainADCChannels]<243.145298s (+0.000046)> brick_id: unsupported
[measureMainADCChannels]<243.145343s (+0.000045)> adc_in7: unsupported
[_ReadGPADCInternal]<243.149830s (+0.004487)> ADC raw value = 871
[_ReadGPADCInternal]<243.152042s (+0.002212)> ADC raw value = 870
[_ReadGPADCInternal]<243.154256s (+0.002214)> ADC raw value = 871
[_ReadGPADCInternal]<243.156466s (+0.002210)> ADC raw value = 871
\end{TerminalOutput}

\subsection{Symbols}

The symbols command is used for symbolication. It automatically detects whether
you are running a local or stock build and symbolicates appropriately.

If you are running a stock build then it retrieves the symbols for that build
from the build server.

To override the symbol and source location a ~\param{workspace}~ option can be given.
The option can be either a local or relative path to the shasta repo. When symbolicating
the new path will be used to locate symbols and also to map the source.

\textcolor{SmokeyDarkBlue}{\bfseries Usage:} \cmdline{symbols [--workspace <path>]}

\textcolor{SmokeyDarkBlue}{\bfseries Sample Output:} \\
\begin{TerminalOutput}
(lldb) symbols
Retrieving symbols from HOB
Loading symbol files to LLDB
File Path Found:/build/archive/shasta_j82_702/***/bootloader/Dwi.macho @ 0x87e3b5000
File Path Found:/build/archive/shasta_j82_702/***/bootloader/Tristar.macho @ 0x87e2c6000
File Path Found:/build/archive/shasta_j82_702/***/bootloader/PMGR.macho @ 0x87e261000
File Path Found:/build/archive/shasta_j82_702/***/bootloader/Gpio.macho @ 0x87e2de000
File Path Found:/build/archive/shasta_j82_702/***/bootloader/DiagBds.macho @ 0x87e2f5000
File Path Found:/build/archive/shasta_j82_702/***/bootloader/ChipId.macho @ 0x87e2ad000
File Path Found:/build/archive/shasta_j82_702/***/bootloader/Timer.macho @ 0x87e2a8000
File Path Found:/build/archive/shasta_j82_702/***/bootloader/Variable.macho @ 0x87e2a4000
File Path Found:/build/archive/shasta_j82_702/***/bootloader/Smbus.macho @ 0x87e243000
File Path Found:/build/archive/shasta_j82_702/***/bootloader/MCA.macho @ 0x87b4f3000
\end{TerminalOutput}

To verify if symbols have been loaded you can use the command 
\textcolor{SmokeyDarkBlue}{\bfseries image list}

Also to verify that you have the correct source checked out you can use the \textcolor{
SmokeyDarkBlue}{\bfseries build\_info} command to compare the tag of the build against 
your source.

\subsubsection{Fallback mechanism} 
If, for some reason when using the symbols command, you get a message saying that the HOB 
hasn’t been loaded or that you are not connected to the HOB, then there is a very good chance 
that the crash occured in Pre-EFI. In that case you can use the \textcolor{SmokeyDarkBlue}
{\bfseries loadpei} command which is explained in \NumNameRef{sec:LoadPei} as a fallback.

\subsubsection{Optimization}
If you have already loaded the symbols once on your local machine then the next time
you want to load them for the same platform you can use the following command:

\textcolor{SmokeyDarkBlue}{\bfseries Usage:} \cmdline{symbols -p <target-platform>}

This command loads the symbols much faster since it doesn't try to locate them again.
Instead it justs loads a pre-defined file which has the symbols from the previous iteration into LLDB!

\textcolor{SmokeyDarkBlue}{\bfseries Sample Output:} \\
\begin{TerminalOutput}
(lldb) symbols -p J82
Executing commands in '/private/tmp/symbols-J82.lldb'.
(lldb) target modules add /Users/Anish/Documents/shasta-intern/***/AppleConsoleRouter.macho
(lldb) target modules load --file AppleConsoleRouter.macho --slide 0x87dc61000
(lldb) target modules add /Users/Anish/Documents/shasta-intern/***/DxeStatusCode.macho
(lldb) target modules load --file DxeStatusCode.macho --slide 0x87dc17000
(lldb) target modules add /Users/Anish/Documents/shasta-intern/***/ScratchRegister.macho
(lldb) target modules load --file ScratchRegister.macho --slide 0x87da68000
\end{TerminalOutput}

\subsection{Build Information}

This commmand prints out all the information about the build which is 
running on the device currently.

\textcolor{SmokeyDarkBlue}{\bfseries Usage:} \cmdline{build\_info}

\textcolor{SmokeyDarkBlue}{\bfseries Sample Output:} \\

\begin{TerminalOutput}
(lldb) build_info
PlatformID: J82
BuildType: BuildEng
BuildTrain: Shasta
DiagsBuildBranch: master
DiagsBuildID: 11D2130
BuildNumber: 702
\end{TerminalOutput}

\subsection{Triage}

The triage command is used mainly by the panic.apple.com server, and is a simple way to run all
of the above commands. It also will dump the first 0x100 bytes of RAM to help diagnose triage
times when false EFI panics are reported.

\textcolor{SmokeyDarkBlue}{\bfseries Usage:} \cmdline{triage}

\textcolor{SmokeyDarkBlue}{\bfseries Sample Output:} \\

\begin{TerminalOutput}
(lldb) triage
Running: memory read --binary -c 0x100 0x80010000
0x80010000: 01 00 38 00 00 00 00 00 09 00 00 00 00 00 00 00  ..8.............
0x80010010: 00 00 00 a0 00 00 00 00 00 00 01 80 00 00 00 00  ...?............
0x80010020: 00 a0 f8 9f 00 00 00 00 a8 04 01 80 00 00 00 00  .??.....?.......
0x80010030: a0 04 01 80 00 00 00 00 05 00 18 00 00 00 00 00  ?...............
0x80010040: 00 10 00 84 00 00 00 00 00 f0 2f 00 00 00 00 00  .........?/.....
0x80010050: 02 00 30 00 00 00 00 00 00 00 00 00 00 00 00 00  ..0.............
0x80010060: 00 00 00 00 00 00 00 00 00 00 fd 9f 00 00 00 00  ..........?.....
0x80010070: 00 00 03 00 00 00 00 00 04 00 00 00 d5 a7 4c 42  ............էLB
0x80010080: 02 00 30 00 00 00 00 00 00 00 00 00 00 00 00 00  ..0.............
0x80010090: 00 00 00 00 00 00 00 00 00 80 fc 9f 00 00 00 00  ..........?.....
0x800100a0: 00 80 00 00 00 00 00 00 04 00 00 00 a6 b9 6c 25  ............??l%
0x800100b0: 04 00 20 00 00 00 00 00 c9 8c d1 1f c5 03 a4 47  .. .....?.?.?.?G
0x800100c0: ba d5 1d 1c a0 81 4e 6d 00 80 fc 9f 77 d2 31 3c  ??..?.Nm..?.w?1<
0x800100d0: 06 00 10 00 00 00 00 00 20 00 d1 bd ab af a2 db  ........ .ѽ????
0x800100e0: 04 00 58 00 00 00 00 00 d2 e7 91 b0 a0 05 98 41  ..X.....??.??..A
0x800100f0: 94 f0 74 b7 b8 c5 54 59 00 00 00 fe 2d 09 e5 d6  .?t???TY...?-.??
Running: build_info
PlatformID: N27
BuildType: Engineering
BuildTrain: Shasta
DiagsBuildBranch: master
DiagsBuildID: 00A0001
BuildNumber: 
SrcRevision: 78696d6
\end{TerminalOutput}

\subsection{Workspace} 
Imagine a situation when you are downloading symbols from the build server, which may be the case when you
are running a stock build on a device. If you want to get detailed information when
executing a backtrace command, you can use the following command to map the source
files on your machine to those which were used to build the stock build on the device:

\textcolor{SmokeyDarkBlue}{\bfseries Usage:} \cmdline{workspace [path-to-shasta]}

\subsection{Load Pre-EFI Symbols}
\label{sec:LoadPei}

To load the Pre-EFI symbols use the loadpei command. This is particularly 
helpful to debug crashes in Pre-EFI state. If the symbols command is successfully
executed prior to running this command then the Pre-EFI symbols would have already
been loaded as a part of its execution.

\textcolor{SmokeyDarkBlue}{\bfseries Usage:} \\
\cmdline{loadpei} \\
\cmdline{loadpei -t <target\_platform>}

\textbf{Note:} Its easy to determine whether the HOB has been initialized by just
running \textcolor{SmokeyDarkBlue}{\bfseries loadpei} initially and if
it says that HOB hasn't been initialized then specify the target platform
along with the command.

\textcolor{SmokeyDarkBlue}{\bfseries Sample Output:} \\
\begin{TerminalOutput}
(lldb) loadpei
Executing commands in '/Users/Anish/Documents/***/load_pei_symbols.lldb'.
(lldb) target modules add /Users/Anish/Documents/***/debug/bootloader/VectorJumpIsland.obj
(lldb) target modules load --file VectorJumpIsland.obj --slide 0x804000000
(lldb) target modules add /Users/Anish/Documents/***/debug/bootloader/SecCore.macho
(lldb) target modules load --file SecCore.macho --slide 0x80436c000
(lldb) target modules add /Users/Anish/Documents/***/debug/bootloader/PreEfi.macho
(lldb) target modules load --file PreEfi.macho --slide 0x8042f8000
\end{TerminalOutput}

\subsection{Hob}

This command prints all the entries in the HOB List.

\textcolor{SmokeyDarkBlue}{\bfseries Usage:} \cmdline{hob -l}

To print just a specific entry in the HOB List

\textcolor{SmokeyDarkBlue}{\bfseries Usage:} \cmdline{hob -t <Name of Entry>}

\textcolor{SmokeyDarkBlue}{\bfseries Sample Output:} \\
\begin{TerminalOutput}
(lldb) hob -l
Hob Type: 0x1, HobLength: 56
Hob Type: 0x5, HobLength: 24
Hob Type: 0x2, HobLength: 48
Hob Type: 0x2, HobLength: 48
Hob Type: 0x2, HobLength: 48
Hob Type: 0x4, HobLength: 32
Hob Type: 0x6, HobLength: 16
Hob Type: 0x2, HobLength: 48
Hob Type: 0x4, HobLength: 56
Hob Type: 0x4, HobLength: 88
Hob Type: 0x4, HobLength: 88
Hob Type: 0x2, HobLength: 48
Hob Type: 0x4, HobLength: 56
Hob Type: 0x4, HobLength: 32
Hob Type: 0x4, HobLength: 32
Hob Type: 0x4, HobLength: 32
Hob Type: 0x4, HobLength: 32
Hob Type: 0x4, HobLength: 32
Hob Type: 0x4, HobLength: 32
Hob Type: 0x3, HobLength: 48
Hob Type: 0x2, HobLength: 48
Hob Type: 0x2, HobLength: 48
Hob Type: 0x2, HobLength: 48
Hob Type: 0x2, HobLength: 48
Hob Type: 0xb, HobLength: 80
Hob Type: 0x9, HobLength: 56
Hob Type: 0xc, HobLength: 24
Hob Type: 0xa, HobLength: 16
Hob Type: 0xffff, HobLength: 8
EFI_HOB_HANDOFF_INFO_TABLE(0x87fb9f598):
	Header:EFI_HOB_GENERIC_HEADER(0x87fb9f598):
		HobType(UINT16):
			0x87fb9f598: 0x0001
		HobLength(UINT16):
			0x87fb9f59a: 0x0038
		Reserved(UINT32):
			0x87fb9f59c: 0x00000000
	Version(UINT32):
		0x87fb9f5a0: 0x00000009
	BootMode(UINT32):
		0x87fb9f5a4: 0x00000000
	EfiMemoryTop(EFI_PHYSICAL_ADDRESS):
		0x87fb9f5a8: 0x0000000880000000
	EfiMemoryBottom(EFI_PHYSICAL_ADDRESS):
		0x87fb9f5b0: 0x0000000800000000
	EfiFreeMemoryTop(EFI_PHYSICAL_ADDRESS):
		0x87fb9f5b8: 0x000000087fba5000
	EfiFreeMemoryBottom(EFI_PHYSICAL_ADDRESS):
		0x87fb9f5c0: 0x00000008000004f8
	EfiEndOfHobList(EFI_PHYSICAL_ADDRESS):
		0x87fb9f5c8: 0x00000008000004f0
EFI_HOB_FIRMWARE_VOLUME(0x87fb9f5d0):
	Header:EFI_HOB_GENERIC_HEADER(0x87fb9f5d0):
		HobType(UINT16):
			0x87fb9f5d0: 0x0005
		HobLength(UINT16):
			0x87fb9f5d2: 0x0018
		Reserved(UINT32):
			0x87fb9f5d4: 0x00000000
	BaseAddress(EFI_PHYSICAL_ADDRESS):
		0x87fb9f5d8: 0x0000000804001000
	Length(UINT64):
		0x87fb9f5e0: 0x000000000030f000


(lldb) hob -t EFI_HOB_SYSTEM_TABLE
EFI_HOB_SYSTEM_TABLE(0x87fb75a78):
	Header:EFI_HOB_GENERIC_HEADER(0x87fb75a78):
		HobType(UINT16):
			0x87fb75a78: 0x000a
		HobLength(UINT16):
			0x87fb75a7a: 0x0010
		Reserved(UINT32):
			0x87fb75a7c: 0x00000000
	TableLocation(EFI_PHYSICAL_ADDRESS):
		0x87fb75a80: 0x000000087fb79f18
\end{TerminalOutput}

\subsection{Configuration Table}

This command prints out the configuration table. It shows you the number
of tables and their respective GUID information.

\textcolor{SmokeyDarkBlue}{\bfseries Usage:} \cmdline{config\_table}

\textcolor{SmokeyDarkBlue}{\bfseries Sample Output:} \\
\begin{TerminalOutput}
(lldb) config_table
EFI_SYSTEM_TABLE(0x87fba3f18):
	Hdr:EFI_TABLE_HEADER(0x87fba3f18):
		Signature(UINT64):
			0x87fba3f18: 0x5453595320494249
		Revision(UINT32):
			0x87fba3f20: 0x0001000a
		HeaderSize(UINT32):
			0x87fba3f24: 0x00000078
		CRC32(UINT32):
			0x87fba3f28: 0xa9ea8f10
		Reserved(UINT32):
			0x87fba3f2c: 0x00000000
	FirmwareVendor(CHAR16*):
		0x87fba3f30: 0x0000000000000000
	FirmwareRevision(UINT32):
		0x87fba3f38: 0x00000000
	ConsoleInHandle(EFI_HANDLE):
		0x87fba3f40: 0x0000000000000000
	ConIn(EFI_SIMPLE_TEXT_INPUT_PROTOCOL*):
		0x87fba3f48: 0x000000087e112000
	ConsoleOutHandle(EFI_HANDLE):
		0x87fba3f50: 0x0000000000000000
	ConOut(EFI_SIMPLE_TEXT_OUTPUT_PROTOCOL*):
		0x87fba3f58: 0x000000087e112030
	StandardErrorHandle(EFI_SIMPLE_TEXT_OUTPUT_PROTOCOL):
		0x87fba3f60: 0x0000000000000000
	StdErr(EFI_SIMPLE_TEXT_OUTPUT_PROTOCOL*):
		0x87fba3f68: 0x0000000000000000
	RuntimeServices(EFI_RUNTIME_SERVICES*):
		0x87fba3f70: 0x000000087fba2d98
	BootServices(EFI_BOOT_SERVICES*):
		0x87fba3f78: 0x000000087fbb2000
	NumberOfTableEntries(UINTN):
		0x87fba3f80: 0x0000000000000004
	ConfigurationTable(EFI_CONFIGURATION_TABLE*):
		0x87fba3f88: 0x000000087fb9ee18
NumberOfTableEntries = 4
ConfigurationTable = 87fb9ee18
05AD34BA-6F02-4214-952E-4DA0398E2BB9  VendorTable = 0x87fbb2170
7739F24C-93D7-11D4-9A3A-0090273FC14D  VendorTable = 0x87fb9f598
4C19049F-4137-4DD3-9C10-8B97A83FFDFA  VendorTable = 0x87fbb2938
49152E77-1ADA-4764-B7A2-7AFEFED95E8B  VendorTable = 0x87fbb58d8
\end{TerminalOutput}

